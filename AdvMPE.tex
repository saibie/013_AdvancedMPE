%% LyX 1.6.4 created this file.  For more info, see http://www.lyx.org/.
%% Do not edit unless you really know what you are doing.
\documentclass[10pt,a4paper,english]{amsart}
\usepackage[T1]{fontenc}
\usepackage[latin9]{inputenc}
%\usepackage{endnotes}
\usepackage{units}
%\usepackage{multirow}
\usepackage{amstext}
\usepackage{amsmath}
\usepackage{amssymb}
\usepackage{amsfonts}
\usepackage{enumerate}
\usepackage{cite}
%\usepackage{natbib}
\usepackage{amsthm}
\usepackage{array,arydshln}
\usepackage[pdftex]{graphicx}
\usepackage{rotating}
\usepackage{ifpdf}
%\usepackage{epsfig}
\usepackage[all]{xy}
\usepackage{latexsym}
\usepackage[hidelinks]{hyperref}
\usepackage{color}
%\usepackage{hfont}


\makeatletter
%%%%%%%%%%%%%%%%%%%%%%%%%%%%%% Textclass specific LaTeX commands.
\numberwithin{equation}{section} %% Comment out for sequentially-numbered
\numberwithin{figure}{section} %% Comment out for sequentially-numbered
\numberwithin{table}{section}
\let\footnote=\endnote
\theoremstyle{plain}
\newtheorem{thm}{Theorem}[section]
\theoremstyle{definition}
\newtheorem{defn}[thm]{Definition}
\newtheorem{Def}[thm]{Definition}
\newtheorem{definition}[thm]{Definition}
\newtheorem{exam}[thm]{Example}
\newtheorem{algo}[thm]{Algorithm}
%  \theoremstyle{plain}
\newtheorem{assumption}[thm]{Assumption}
\theoremstyle{plain}
\newtheorem{lem}[thm]{Lemma}
\newtheorem{lemma}[thm]{Lemma}
\theoremstyle{plain}
\newtheorem{cor}[thm]{Corollary}
\newtheorem{corollary}[thm]{Corollary}
\theoremstyle{plain}
\newtheorem{rmk}[thm]{Remark}
\newtheorem{rem}[thm]{Remark}

\def\norm#1{\|#1\|}

\newcommand\numberthis{\addtocounter{equation}{1}\tag{\theequation}}

%\newcommand{\norm}[1]{\|#1\|}
\def\norm#1{\|#1\|}
\def\normm#1#2{\|#1\|_{#2}}
\def\normF#1{\|#1\|_{F}}
\def\Proof{{\bf Proof.\enspace}}
\def\vec{\mathrm{vec}}
%\def\unvec{\mathrm{unvec}}
\def\tr{\mathrm{tr}}
%\def\tr{\textrm{tr}}
\def\bmatrix#1{\left[\begin{matrix}#1\end{matrix}\right]}
\def\pmatrix#1{\left(\begin{matrix}#1\end{matrix}\right)}
\def\R{\mathbb{R}}
\def\N{\mathbb{N}}
\def\C{\mathbb{C}}
\def\nbyn{n\times n}
\def\mbyn{m\times n}
\def\mbym{m\times m}
\def\pbyq{p\times q}
\def\nnbynn{n^{2}\times n^{2}}
\def\mbf#1{\mathbf{#1}}
\def\mrm#1{\mathrm{#1}}
\def\bpi{\boldsymbol{\pi}}
\def\a{\alpha}
\def\b{\beta}
\def\d{\delta}
\def\e{\varepsilon}
\def\l{\lambda}
\def\D{\mathcal{D}}
\def\F{\mathcal{F}}
\def\G{\mathcal{G}}
\def\Q{\mathcal{Q}}
\def\M{\mathcal{M}}
\def\P{\mathcal{P}}
\def\X{\mathcal{X}}
\def\pjn{\mathbf{P}_{\mathcal{N}}}
\def\pjm{\mathbf{P}_{\mathcal{M}}}
\def\tXi{\tilde{X}_{i}}
\def\tXii{\tilde{X}_{i+1}}
\def\dpm#1{\begin{displaymath}#1\end{displaymath}}
\def\bdm{\begin{displaymath}}
\def\edm{\end{displaymath}}
\def\dtyl{\displaystyle}
\def\ones#1{\mathbf{1}_{#1}}
%\def\onesn{\mathbf{1}_{n \times n}}

\definecolor{gray}{rgb}{.5,.5,.5}

\makeatother

\begin{document}
	
\title[Solving Nearly Non-simple MPE by Newton's Method with Line Searches]{Solving Nearly Non-simple Matrix Polynomial Equations by Newton's Method with Advanced Line Searches}
\author{Sang-hyup Seo}
\address{Sang-hyup Seo\\Where}
\email{saibie1677@gmail.com}
\date{\today}

\begin{abstract}
%	We consider the Newton iteration for a matrix polynomial equation which arises in stochastic problem.
%	In this paper, we show that the elementwise minimal nonnegative solution of the matrix polynomial equation can be obtained using Newton's method if the equation satisfies the sufficient condition, and
%	the convergence rate of the iteration is quadratic if the solution is simple.
%	Moreover, we show that the convergence rate is at least linear if the solution is non-simple,
%	but we can apply a modified Newton method whose iteration number is less than the pure Newton iteration number.
%	Finally, we give a numerical experiment which is related with our issue.
\end{abstract}

\keywords{matrix polynomial equation, elementwise positive solution, elementwise nonnegative solution, $M$-matrix, Newton's method, line search, acceleration of a method, nearly non-simple}

\subjclass[2010]{65H10}

% \thanks{$\dagger$Corresponding Author}

\maketitle

\section{Introduction}\label{sec:intro}

We consider a matrix polynomial equation(MPE) with $n$-degree defined by
\begin{equation}\label{eq:MPE}
P(X)=\sum_{k=0}^{n} A_{k}X^{k}= A_{n}X^{n}+A_{n-1}X^{n-1}+\cdots+A_{1}X + A_{0}=0,
\end{equation}
where the coefficient matrices $A_{k}$'s are $\mbym$ matrices.
Then, the unknown matrix $X$ must be an $\mbym$ matrix.


The MPE \eqref{eq:MPE} often occurs in the theory of differential equations, system theory, network theory, stochastic theory, quasi-birth-and-death
and other areas \cite{ALFA2003, Bean1997, Butler1985, Gohberg1982, Lancaster1966, Lancaster1985, He2001, bini2005, Latouche1999}.
Specially, in quasi-birth-and-death and stochastic problems, finding the minimal nonnegative solution of a matrix equation is an important issue.

There are many researches to find the minimal nonnegative solution.
%Davis \cite{Davis1981, Davis1983} and Higham, Kim \cite{Higham2000, Higham2001} studied the Newton method for a quadratic matrix equation.
Guo and Laub \cite{Guo2000} considered a nonsymmetric algebraic Riccati equation, and they proposed iteration algorithms which converge to the minimal positive solution. 
In \cite{Guo2001}, Guo provided a sufficient condition for the existence of nonnegative solutions of nonsymmetric algebraic Riccati equations.
Kim \cite{Kim2008} showed that the minimal positive solutions also can be found by the Newton method with the zero initial matrices in some different types of quadratic equations.
%Hautphenne, Latouche, and Remiche \cite{SophieHautphenne2008} studied the Newton method for the Markovian binary tree.
Seo and Kim \cite{SeoSeoKim2013, SeoKim2014} studied the Newton iteration for a quadratic matrix equation and a matrix polynomial equation.

Newton's method is one of powerful tools to find solutions of nonlinear matrix equations.
By Kantorovich theorem \cite{Kantorovich1964}, the convergence rate of the method is quadratic if the derivative on the domain is Lipschitz continuous and at the solution is nonsingular.
But, if the derivative at the solution is singular, then we cannot apply Kantrovich theorem, i.e., we cannot guarantee that the rate is quadratic.
The followings are researches to analyze the problems with singular derivative at the solution and improve the method.

For general functions on Banach spaces, Reddien \cite{Reddien1978}, Decker and Kelley \cite{DeckerKelley1980Newton1, DeckerKelley1980Newton2} gave analyses about Newton's method for singular problems.
In \cite{DeckerKellerKelley1983}, Decker, Keller, and Kelley provided an acceleration of Newton's method for singular problems and analyzed for the convergence rate of the method.
Kelley and Suresh \cite{Kelley1983} suggested a new accelerated Newton's method at singular points.
Decker and Kelley \cite{DeckerKelley1985} showed an analysis for Newton's method
at nearly singular roots.
In \cite{Kelley1986}, Kelley analyzed convergence rate of the method for functions whose high order derivatives at the solution are singular.
For specific functions on $\R^{\mbyn}$, Guo and Lancaster \cite{Guo1998M} analyzed and provided a modification about Newton's method for algebraic Riccati equations at singular roots.
In \cite{SeoSeo2020}, S-.H. Seo and J-.H. Seo suggested a modified Newton method for matrix polynomial equations with $n$-degree for singular problems.

In \cite{Kelley1983, DeckerKellerKelley1983}, accelerations for Newton's method at singular roots was suggested for general functions on Banach spaces.
For specific functions on $\R^{\mbyn}$, modifications of Newton's method at singular roots was provided in \cite{Guo1998M, SeoSeoKim2018}.
But, the main hypotheses of the papers are similar and that the solution is non-simple, i.e., the derivative at the solution is singular.
It means that the accelerations for Newton's method cannot be applied and the accelerated iterations cannot be guaranteed converge to the solutions.
Even if the iterations converge to the solutions, they cannot be guaranteed faster than the pure iterations.


Otherwise, in \cite{DeckerKelley1985}, the authors analyzed and suggested an acceleration for Newton's method about the case of nearly singular roots.

We give some basic definitions and lemmas for this paper.

Let $A, B \in \R^{\mbym}$ be matrices.
If all elements of $A$ is nonnegative, then we call that $A$ is a {\it nonnegative} matrix and denote $A \geq 0$. In similar sense, we define $A \leq 0$, $A > 0$, and $A < 0$.
If a matrix $A$ can be written as $rI-B$ with $B \geq 0$ and $r \in \R$, we call that $A$ is a \textit{$Z$-matrix}.
Moreover, if $A=rI-B$ is a $Z$-matrix and $r\geq\rho(B)$ then $A$ is called an $M$-matrix.
The following is a basic theorem for $M$-matrices.

\begin{thm}\label{thm:proMmat}%{\rm \cite[Theorem 2.1]{Guo2007}, \cite[Theorem 2.1]{Poole1974}}
	For a $Z$-matrix $A$, the following are equivalent:
	\begin{enumerate}
		\item $A$ is a nonsingular $M$-matrix.
		
		\item $A^{-1}$ is nonnegative.
		
		\item $Av>0$ for some vector $v>0$.
		
		\item All eigenvalues of $A$ have positive real parts.
	\end{enumerate}
\end{thm}

Let a function $F : \R^{\mbyn} \rightarrow \R^{\mbyn}$ and an equation $F(X) = 0$ be given.

If nonnegative solutions $S_{1}$ and $S_{2}$ of $F(X) = 0$ are satisfy
	\begin{equation}\label{eq:s1ss2}
	S_{1}\leq S\leq S_{2},\end{equation}
for any nonnegative solution $S$ of $F(X) = 0$, then they are called the {\it minimal nonnegative solution} and the {\it maximal nonnegative solution}, respectively.
The {\it minimal positive solution} and the {\it maximal positive solution} also are can be defined, similarly.
If the Fr\'echet derivative of $F$ at a solution $S$ is nonsingular, then $S$ is called {\it simple}.
Furthermore, we call that a not simple solution is {\it non-simple}.

\begin{assumption}\label{ass:MPE}
	For the MPE \eqref{eq:MPE},
	\begin{enumerate}[1)]
		\item The coefficient matrices $A_{k}$'s are nonnegative except $A_{1}$.
		\item $-A_{1}$ is a nonsingular $M$-matrix.
		\item $\sum_{k=0}^{n}A_{k}$ is irreducible.
		\item $A_{0}$, $A_{1}$, and $\sum_{k=2}^{n}A_{k}$ are irreducible.
	\end{enumerate}
\end{assumption}



%Aim of this paper is to provide an improved Newton's method for \eqref{eq:MPE} at nearly singular roots.


For convenience, the notation $||\,\cdot\,||$ is used instead of the Frobenius norm $||\,\cdot\,||_{F}$
and $\N_{0}$ is used as $\N \cup \{0\}$ because the Frobenius norm and $\N_{0}$ are used very frequently in this paper.




\bibliographystyle{plain}
\bibliography{SHSeo}

\end{document}