%% LyX 1.6.4 created this file.  For more info, see http://www.lyx.org/.
%% Do not edit unless you really know what you are doing.
\documentclass[10pt,a4paper,english]{amsart}
\usepackage[T1]{fontenc}
\usepackage[latin9]{inputenc}
%\usepackage{endnotes}
\usepackage{units}
%\usepackage{multirow}
\usepackage{amstext}
\usepackage{amsmath}
\usepackage{amssymb}
\usepackage{amsfonts}
\usepackage{enumerate}
\usepackage{cite}
%\usepackage{natbib}
\usepackage{amsthm}
\usepackage{array,arydshln}
\usepackage[pdftex]{graphicx}
\usepackage{rotating}
\usepackage{ifpdf}
%\usepackage{epsfig}
\usepackage[all]{xy}
\usepackage{latexsym}
\usepackage[hidelinks]{hyperref}
\usepackage{color}
%\usepackage{hfont}


\makeatletter
%%%%%%%%%%%%%%%%%%%%%%%%%%%%%% Textclass specific LaTeX commands.
\numberwithin{equation}{section} %% Comment out for sequentially-numbered
\numberwithin{figure}{section} %% Comment out for sequentially-numbered
\numberwithin{table}{section}
\let\footnote=\endnote
\theoremstyle{plain}
\newtheorem{thm}{Theorem}[section]
\theoremstyle{definition}
\newtheorem{defn}[thm]{Definition}
\newtheorem{Def}[thm]{Definition}
\newtheorem{definition}[thm]{Definition}
\newtheorem{exam}[thm]{Example}
\newtheorem{algo}[thm]{Algorithm}
%  \theoremstyle{plain}
\newtheorem{assumption}[thm]{Assumption}
\theoremstyle{plain}
\newtheorem{lem}[thm]{Lemma}
\newtheorem{lemma}[thm]{Lemma}
\theoremstyle{plain}
\newtheorem{cor}[thm]{Corollary}
\newtheorem{corollary}[thm]{Corollary}
\theoremstyle{plain}
\newtheorem{rmk}[thm]{Remark}
\newtheorem{rem}[thm]{Remark}

\def\norm#1{\|#1\|}

\newcommand\numberthis{\addtocounter{equation}{1}\tag{\theequation}}

%\newcommand{\norm}[1]{\|#1\|}
\def\norm#1{\|#1\|}
\def\normm#1#2{\|#1\|_{#2}}
\def\normF#1{\|#1\|_{F}}
\def\Proof{{\bf Proof.\enspace}}
\def\vec{\mathrm{vec}}
%\def\unvec{\mathrm{unvec}}
\def\tr{\mathrm{tr}}
%\def\tr{\textrm{tr}}
\def\bmatrix#1{\left[\begin{matrix}#1\end{matrix}\right]}
\def\pmatrix#1{\left(\begin{matrix}#1\end{matrix}\right)}
\def\R{\mathbb{R}}
\def\N{\mathbb{N}}
\def\C{\mathbb{C}}
\def\nbyn{n\times n}
\def\mbyn{m\times n}
\def\mbym{m\times m}
\def\pbyq{p\times q}
\def\nnbynn{n^{2}\times n^{2}}
\def\mbf#1{\mathbf{#1}}
\def\mrm#1{\mathrm{#1}}
\def\bpi{\boldsymbol{\pi}}
\def\a{\alpha}
\def\b{\beta}
\def\d{\delta}
\def\e{\varepsilon}
\def\l{\lambda}
\def\D{\mathcal{D}}
\def\F{\mathcal{F}}
\def\G{\mathcal{G}}
\def\Q{\mathcal{Q}}
\def\M{\mathcal{M}}
\def\P{\mathcal{P}}
\def\X{\mathcal{X}}
\def\pjn{\mathbf{P}_{\mathcal{N}}}
\def\pjm{\mathbf{P}_{\mathcal{M}}}
\def\tXi{\tilde{X}_{i}}
\def\tXii{\tilde{X}_{i+1}}
\def\dpm#1{\begin{displaymath}#1\end{displaymath}}
\def\bdm{\begin{displaymath}}
\def\edm{\end{displaymath}}
\def\dtyl{\displaystyle}
\def\ones#1{\mathbf{1}_{#1}}
%\def\onesn{\mathbf{1}_{n \times n}}

\definecolor{gray}{rgb}{.5,.5,.5}

\makeatother

\begin{document}
	
\title{AdvancedMPE}
\author{Sang-hyup Seo}
\address{Sang-hyup Seo\\Where}
\email{saibie1677@gmail.com}
\date{\today}

\begin{abstract}
	We consider the Newton iteration for a matrix polynomial equation which arises in stochastic problem.
	In this paper, we show that the elementwise minimal nonnegative solution of the matrix polynomial equation can be obtained using Newton's method if the equation satisfies the sufficient condition, and
	the convergence rate of the iteration is quadratic if the solution is simple.
	Moreover, we show that the convergence rate is at least linear if the solution is non-simple,
	but we can apply a modified Newton method whose iteration number is less than the pure Newton iteration number.
	Finally, we give a numerical experiment which is related with our issue.
\end{abstract}

\keywords{matrix polynomial equation, elementwise positive solution, elementwise nonnegative solution, $M$-matrix, Newton's method, convergence rate, acceleration of a method}

\subjclass[2010]{65H10}

% \thanks{$\dagger$Corresponding Author}

\maketitle

\section{Introduction}\label{sec:intro}

We consider a matrix polynomial equation(MPE) with $n$-degree defined by
\begin{equation}\label{eq:MPE}
P(X)=\sum_{k=0}^{n} A_{k}X^{k}= A_{n}X^{n}+A_{n-1}X^{n-1}+\cdots+A_{1}X + A_{0}=0,
\end{equation}
where the coefficient matrices $A_{k}$'s are $\mbym$ matrices.
Then, the unknown matrix $X$ must be an $\mbym$ matrix.


The MPE \eqref{eq:MPE} often occurs in the theory of differential equations, system theory, network theory, stochastic theory, quasi-birth-and-death
and other areas \cite{ALFA2003, Bean1997, Butler1985, Gohberg1982, Lancaster1966, Lancaster1985, He2001, bini2005, Latouche1999}.

Davis \cite{Davis1981, Davis1983} and Higham, Kim \cite{Higham2000, Higham2001} studied the Newton method for a quadratic matrix equation.
Guo and Laub \cite{Guo2000} considered a nonsymmetric algebraic Riccati equation, and they proposed iteration algorithms which converge to the minimal positive solution. In \cite{Guo2001}, Guo provided a sufficient condition for the existence of nonnegative solutions of nonsymmetric algebraic Riccati equations.
Kim \cite{Kim2008} showed that the minimal positive solutions also can be found by the Newton method with the zero initial matrices 
in some different types of quadratic equations.
Hautphenne, Latouche, and Remiche \cite{SophieHautphenne2008} studied the Newton method for the Markovian binary tree.


Seo and Kim \cite{SeoSeoKim2013, SeoKim2014} studied the Newton iteration for a quadratic matrix equation and a matrix polynomial equation.
Specially, in \cite{SeoKim2014}, they provided a relaxed Newton method whose convergence is faster than the pure one.
Guo and Lancaster \cite{Guo1998M} analyzed and provided a modification about Newton's method for algebraic Riccati equations.
They showed that the modification of Newton's method is better than the pure one 
if the minimal nonnegative solution is non-simple.

\begin{assumption}\label{ass:MPE}
	For the MPE \eqref{eq:MPE},
	\begin{enumerate}[1)]
		\item The coefficient matrices $A_{k}$'s are nonnegative except $A_{1}$.
		\item $-A_{1}$ is a nonsingular $M$-matrix.
		\item $\sum_{k=0}^{n}A_{k}$ are irreducible.
	\end{enumerate}
\end{assumption}


Our goal is to propose a singular escaping Newton method for the MPE \eqref{eq:MPE} which satisfies Assumption \ref{ass:MPE}.
This MPE is useful for stochastic theory, quasi-birth-and-death problem, and so on.
%The advanced Newton method is better than the pure Newton's method
%if the elementwise minimal positive solution is non-simple.
%The idea of the singular escaping Newton method is from the modification of Newton's method for algebraic Riccati equtions and matrix polynomial equations of \cite{Guo1998M, SeoSeoKim2018}.
In \cite{Guo1998M}, Guo and Lancaster showed that $\|Y_{i+1} - S\| < c\varepsilon$ 
for the modified iteration $Y_{i+1}$, the solution $S$, a constant $c > 0$, and small $\varepsilon > 0$.
Similarily, Seo, Seo, and Kim \cite{SeoSeoKim2018} showed that the modified Newton iteration $Y_{i+1}$ for the MPE is closer to the solution $S$ than
the pure Newton iteration $X_{i+1}$.
But, in both of \cite{Guo1998M, SeoSeoKim2018}, the authors showed that the modifications are better than the pure if the solution $S$ is non-simple.

We start with some basic definitions.
\begin{defn}\label{def:Zmat}%\cite{Guo2007}, \cite[p. 42]{Young1971}
	Let a matrix $A \in \R^{\mbym}$.
	$A$ is an \textit{$Z$-matrix} if all its off-diagonal elements
	are nonpositive.
\end{defn}

It is clear that any $Z$-matrix $A$ can be written as $sI-B$ with $B\geq0$ and $s \in \R$.
Then $M$-matrix can be defined as follows.

\begin{defn}\label{def:Mmat}%\cite[p. 580]{Guo2007}
	A matrix $A\in\mathbb{R}^{m\times m}$
	is an \textit{$M$-matrix} if $A=rI-B$ for some nonnegative matrix $B$
	with $r\geq\rho(B)$ where $\rho$ is the spectral radius; it is a
	singular $M$-matrix if $r=\rho(B)$ and a nonsingular $M$-matrix
	if $r>\rho(B)$.
\end{defn}

The following result is well known and can be found in \cite{Guo2007} and \cite{Poole1974} for example.

\begin{thm}\label{thm:proMmat}%{\rm \cite[Theorem 2.1]{Guo2007}, \cite[Theorem 2.1]{Poole1974}}
	For a $Z$-matrix $A$, the following are equivalent:
	\begin{enumerate}
		\item $A$ is a nonsingular $M$-matrix.
		
		\item $A^{-1}$ is nonnegative.
		
		\item $Av>0$ for some vector $v>0$.
		
		\item All eigenvalues of $A$ have positive real parts.
	\end{enumerate}
\end{thm}


%\begin{lem}\label{lem:MtimesM}
%Let $A$ and $B$ be $M$-matrices with same sizes.
%If $AB$ is a $Z$-matrix, then $AB$ is an $M$-matrix.
%\end{lem}
%
%\begin{proof}
%Suppose that $A$ and $B$ are nonsingular.
%Then, by Theorem \ref{thm:proMmat}, $AB$ is a nonsingular $M$-matrix since $(AB)^{-1} = B^{-1}A^{-1}$ is nonnegative.
%
%Suppose that $A$ and $B$ are singular.
%Put $A = r(I - N)$ and $B = s(I - M)$ where $N$ and $M$ are nonnegative matrices such that $\rho(N) = \rho(M) = 1$.
%We will show that $\rho(N + M - NM) \leq 1$.
%For any $\e ,~ \d >0$,
%$$A_{\e} = (1+\e)I - N \qquad \textrm{and} \qquad B_{\d} = (1+\d)I - M$$
%are nonsingular $M$-matrices.
%So,
%$$A_{\e}B_{\d} = (1+\e + \d + \e \d)I - ((1+\d)N + (1+ \e)M - NM)$$
%is a nonsingular $M$-matrix.
%It means that $\rho((1+\d)N + (1+\e)M - NM) < 1+ \e + \d + \e \d$.
%Suppose that $\rho(N + M - NM) > 1$ and put $\gamma = \rho(N + M - NM) - 1$.
%Since $0 \leq N + M - NM \leq (1+\d)N + (1+\e)M - NM$ for all $\e,~ \d > 0$, $\rho(N + M - NM) \leq \rho((1+\d)N + (1+\e)M - NM)$
%by \cite[Theorem 8.1.18]{Horn1985}.
%It's a contradiction.
%Therefore, $\rho(N + M - NM) \leq 1$, i.e., $AB$ is an $M$-matrix.
%\end{proof}




\begin{defn}\label{def:Maxminsolution}
	A positive solution $S_{1}$ of the matrix equation $P(X) = 0$ is
	the \textit{elementwise minimal positive solution} and a positive solution
	$S_{2}$ of $P(X) = 0$ is the \textit{elementwise maximal positive solution}
	if, for any positive solution $S$ of $P(X)$,
	\begin{equation}\label{eq:s1ss2}
	S_{1}\leq S\leq S_{2}.\end{equation}
	
	Similarly, if nonnegative solutions $S_{1}$ and $S_{2}$ satisfy \eqref{eq:s1ss2} for any nonnegative solution $S$,
	then $S_{1}$ is called the \textit{elementwise minimal nonnegative solution} and $S_{2}$ is called the \textit{elementwise maximal nonnegative solution}.
\end{defn}


\begin{defn}{\rm \cite[Definition 4.2.1, Definition 4.2.9]{roger91}}\label{def:KroVec}
	The {\it Kronecker product} of $A = [a_{ij}] \in \C^{m \times n}$ and $B = [b_{ij}] \in \C^{p \times q}$ is denoted by
	$A \otimes B$ and is defined to be the block matrix
	\begin{equation*}
		A \otimes B = \bmatrix{a_{11}B & \cdots & a_{1n}B \\ \vdots & \ddots & \vdots \\ a_{m1}B & \cdots & a_{mn}B} \in \C^{mp \times nq}.
	\end{equation*}
	
	The vec operator $\vec : \C^{m \times n} \rightarrow \C^{mn}$ is defined by
	\begin{equation*}
		\vec(A) = \bmatrix{\mathbf{a}_{1}^{T} & \mathbf{a}_{2}^{T} & \cdots & \mathbf{a}_{n}^{T}}^{T},
	\end{equation*}
	where $\mathbf{a}_{i}^{T} = \bmatrix{a_{1i} & a_{2i} & \cdots & a_{ni}}$.
\end{defn}


\begin{lem}{\rm \cite[Lemma 4.3.1]{roger91}}\label{lem:AXBC}
	Let $A \in \C^{m \times n}, B \in \C^{p \times q},$ and $C \in \C^{m \times q}$ be given
	and let $X \in \C^{n \times p}$ be unknown. The matrix equation
	\begin{equation}
	AXB = C
	\end{equation}
	is equivalent to the system of $qm$ equations in $np$ unknowns given by
	\begin{equation}
	(B^{T} \otimes A) \vec(X) = \vec(C),
	\end{equation}
	that is, $\vec(AXB) = (B^{T} \otimes A)\vec(X)$.
\end{lem}

\begin{defn}\label{def:simple}
	Let a matrix function $F:\C^{\mbyn}\rightarrow\C^{\mbyn}$ be given, and let a matrix equation
	\begin{equation}\label{eq:simple}
	F(X) = 0
	\end{equation}
	be given. Then, a solution $S \in \C^{\mbyn}$ of \eqref{eq:simple} is called {\it simple} if
	the Fr\'{e}chet derivative at $S$ is nonsingular.
	For convenience, a solution is called {\it non-simple} if it is not simple.
\end{defn}






%To reach our goal, in Section \ref{sec:analysis}, we study the minimal nonnegative solution $S$ of \eqref{eq:MPE}
%and the Fr\'{e}chet derivative of $P$ in \eqref{eq:MPE} at $S$,
%and we show the convergence of the Newton iteration for \eqref{eq:MPE}.
%We give an analysis about Newton's method for the non-simple minimal nonnegative solution $S$, in Section \ref{sec:singular}.
%In Section \ref{sec:modified}, we propose a modified Newton method which is better
%for finding the minimal nonnegative solution $S$.
%Finally, we give some numerical experiments, in Section \ref{sec:experiment}.


For convenience, the notation $||\,\cdot\,||$ is used instead of the Frobenius norm $||\,\cdot\,||_{F}$
and $\N_{0}$ is used as $\N \cup \{0\}$ because the Frobenius norm and $\N_{0}$ are used very frequently in this paper.



%\section{Convergence of Newton's Method for an MPE}\label{sec:analysis}
%
%In this section, we introduce a sufficient condition of the existence of
%the minimal nonnegative solution of the MPE \eqref{eq:MPE} with Assumption \ref{ass:MPE},
%and give some analysis for Newton's method.
%
%\begin{thm}\label{thm:SuffMPE}
%{\rm \cite[Theorem 2.1]{Meng2017}}
%Let the MPE \eqref{eq:MPE} with 1) and 2) in Assumption \ref{ass:MPE} be given.
%Then, there exists the minimal nonnegative solution if
%\begin{equation}\label{eq:SuffMPE}
%B = -\sum_{k=0}^{n}A_{k}~\textrm{is a nonsingular or singular irreducible $M$-matrix.}
%\end{equation}
%
%\end{thm}
%
%
%%\begin{proof}
%%Let a matrix function
%%\begin{equation}\label{eq:MPEfix}
%%G(X) = -A_{1}^{-1}\left( \sum_{k=2}^{n}A_{k}X^{k} + A_{0} \right)
%%\end{equation}
%%be given.
%%Consider the sequence $\{X_{i}\}_{i=0}^{\infty}$ defined by
%%\begin{equation}\label{eq:FixedPtEq}
%%X_{i+1} = G(X_{i}),
%%\end{equation}
%%with $X_{0} = 0$.
%%
%%By \cite[Theorem A.16, Theorem A.19]{iannazzo2012}, there exists a vector $v>0$ such that
%%$Bv > 0$ if $B$ is a nonsingular $M$-matrix
%%or $Bv = 0$ if $B$ is a singular irreducible $M$-matrix, i.e.,
%%\begin{equation}
%%\left( -\sum_{k=0}^{n}A_{k} \right) v \geq 0.
%%\end{equation}
%%Since $-A_{1}$ is a nonsingular $M$-matrix, we obtain that
%%\begin{equation}\label{eq:vgeqzero}
%%v \geq -A_{1}^{-1}\left(\sum_{k=2}^{n} A_{k} + A_{0}\right)v \geq 0.
%%\end{equation}
%%
%%
%%We will show that
%%\begin{equation}\label{eq:SufficientInduction}
%%X_{i} \leq X_{i+1} \qquad \textrm{and} \qquad X_{i}v < v,
%%\end{equation}
%%for all $i \in \N_{0}$ using the mathematical induction.
%%
%%Clearly,
%%$$X_{1} = -A_{1}^{-1}A_{0} \geq 0 = X_{0} \qquad \textrm{and} \qquad X_{0}v = 0 < v.$$
%%Hence, \eqref{eq:SufficientInduction} holds for $i = 0$.
%%
%%Suppose that \eqref{eq:SufficientInduction} holds for $i = l$.
%%Then,
%%\begin{equation}
%%X_{l+2} - X_{l+1} = -A_{1}^{-1} \sum_{k=2}^{n}A_{k}(X_{l+1}^{k} - X_{l}^{k}) \geq 0.
%%\end{equation}
%%On the other hand, from \eqref{eq:vgeqzero},
%%$$
%%X_{l+1}v = -A_{1}^{-1}\left(\sum_{k=2}^{n}A_{k}X_{l}^{k} + A_{0}\right)v < -A_{1}^{-1}\left(\sum_{k=2}^{n} A_{k} + A_{0}\right)v \leq v.
%%$$
%%So, \eqref{eq:SufficientInduction} holds for $i = l+1$.
%%By the mathematical induction, \eqref{eq:SufficientInduction} holds for all $i \in \N_{0}$,
%%i.e., $\{X_{i}\}$ converges to a nonnegative matrix.
%%
%%Let $\{X_{i}\}$ converge to a nonnegative matrix $S$ and let $Y$ be a nonnegative solution of \eqref{eq:MPE}.
%%$X_{0} \leq Y$, clearly.
%%Suppose that $X_{l} \leq Y$. Then,
%%$$
%%\begin{array}{rl}
%%Y - X_{l+1} & = \dtyl -A_{1}^{-1}\left(\sum_{k=2}^{n}A_{k}Y^{k} + A_{0}\right) + A_{1}^{-1}\left(\sum_{k=2}^{n}A_{k}X_{l}^{k} + A_{0}\right)\\
%%            & = \dtyl -A_{1}^{-1} \sum_{k=2}^{n} A_{k}(Y^{k} - X_{l}^{k}) \geq 0. \end{array}
%%$$
%%By induction, $X_{i} \leq Y$ for all $i \in \N_{0}$.
%%Therefore, $S \leq Y$ for any nonnegative solution $Y$ of \eqref{eq:MPE},
%%i.e., $S$ is the minimal nonnegative solution of \eqref{eq:MPE}.
%%\end{proof}
%
%
%%\begin{thm}\label{thm:limsupeqrho}
%%Let \eqref{eq:MPE} be given with Assumption \ref{ass:MPE}, and \eqref{eq:SuffMPE},
%%and let $S$ be the minimal nonnegative solution of \eqref{eq:MPE}.
%%Then,
%%\begin{equation}\label{eq:limsupeqrho}
%%\limsup_{i\rightarrow\infty} \sqrt[i]{\|X_{i} - S\|} = \rho(\G'_{S}),
%%\end{equation}
%%where $G'_{X}$ is the Fr\'{e}chet derivative of 
%%\begin{equation}\label{eq:MPEfix}
%%G(X) = -A_{1}^{-1}\left( \sum_{k=2}^{n}A_{k}X^{k} + A_{0} \right)
%%\end{equation}
%%at $X$ and $\G'_{X} = \vec \circ G'_{X} \circ \vec^{-1}$.
%%\end{thm}
%%
%%\begin{proof}
%%Consider the fixed point iteration generated by
%%\begin{equation}
%%X_{i+1} = G(X_{i}),
%%\end{equation}
%%with $X_{0} = -A_{1}^{-1}A_{0}$.
%%According to \cite[Corollary 2.6]{SeoSeo2016}, we need to show that
%%\begin{enumerate}[i)]
%%\item $\{X_{i}\}$ is monotone nondecreasing and converges to $S$,
%%\item $S - X_{0} > 0$,
%%\item $\G'_{X_{0}}$ is a nonnegative matrix such that $\G'_{X_{0}} \ones{} > 0$,
%%\item $\{ \G'_{X_{i}} \}$ is monotone nondecreasing.
%%\end{enumerate}
%%By Theorem \ref{thm:SuffMPE}, $\{X_{i}\}$ converges to $S$.
%%Since $A_{k}$ is irreducible for all $k \in \N_{0}$, $X_{0} > 0$ and
%%\begin{align*}
%%X_{1} &= G(X_{0}) \\
%%      &= -A_{1}^{-1}\left( \sum_{k=2}^{n} A_{k}(-A_{1}^{-1}A_{0})^{k} \right) - A_{1}^{-1}A_{0} \numberthis \\
%%      &> -A_{1}^{-1}A_{0} = X_{0}.
%%\end{align*}
%%Suppose that $X_{i+1} > X_{i}$. Then,
%%\begin{align*}
%%X_{i+2} - X_{i+1} &= -A_{1}^{-1}\left( \sum_{k=2}^{n} A_{k}X_{i+1}^{k} + A_{0} - \sum_{k=2}^{n} A_{k}X_{i}^{k} - A_{0} \right) \\
%%                  &= -A_{1}^{-1}\left( \sum_{k=2}^{n} A_{k}(X_{i+1}^{k} - X_{i}^{k}) \right) > 0.
%%\end{align*}
%%Hence, i) and ii) hold. %$\{X_{i}\}_{i=0}^{\infty}$ is monotone increasing.
%%
%%By \cite[Lemma 4.3.1]{roger91},
%%\begin{equation}
%%\G'_{X_{0}} = \sum_{k=2}^{n} \sum_{l=0}^{k-1} (X_{0}^{k-l-1})^{T} \otimes (-A_{1}^{-1}A_{k}X_{0}^{l}),
%%\end{equation}
%%and is a positive $m^2 \times m^2$ matrix.
%%Moreover, $\{\G'_{X_{i}}\}$ is monotone nondecreasing from the following inequality,
%%\begin{align*}
%%\G'_{X_{i+1}} &= \sum_{k=2}^{n} \sum_{l=0}^{k-1} (X_{i+1}^{k-l-1})^{T} \otimes (-A_{1}^{-1}A_{k}X_{i+1}^{l}) \\
%%                            &\geq \sum_{k=2}^{n} \sum_{l=0}^{k-1} (X_{i}^{k-l-1})^{T} \otimes (-A_{1}^{-1}A_{k}X_{i}^{l})  \numberthis \label{eq:increaseDerivative}\\
%%                   &= \G'_{X_{i}},
%%\end{align*}
%%i.e., iii) and iv) hold.
%%Therefore, the equality \eqref{eq:limsupeqrho} holds.
%%\end{proof}
%%
%%\section{Convergence of Newton's Method}\label{sec:Newton}
%
%The Fr\'{e}chet derivative of the matrix polynomial equation \eqref{eq:MPE} at $X$ in the direction $H$ is given by
%\begin{equation}\label{eq:DerivativeMPE}
%P'_{X}(H)=\sum_{k=1}^{n} \sum_{l=0}^{k-1} A_{k}X^{l}HX^{k-l-1}.
%\end{equation}
%
%The second Fr\'{e}chet derivative of the quadratic
%matrix equation \eqref{eq:MPE} at $X$ is given by
%\begin{equation}\label{eq:2DerivativeMPE}
%P_{X}''(K,H)=\sum_{k=2}^{n}\sum_{l=0}^{k-2}\sum_{j=0}^{l} A_{k} \left( X^{l}HX^{j}KX^{n-l-j-2} + X^{l}KX^{j}HX^{n-l-j-2} \right).
%\end{equation}
%
%
%For the equation \eqref{eq:MPE}, each step of the Newton iteration with given $X_{0}$ can be written as
%\begin{equation}\label{eq:NewtonStepMPE2}
%X_{i+1}=X_{i}-P_{X_{i}}^{\prime -1}(P(X_{i}))
%\end{equation}
%if $P'_{X_{i}}$ is invertible for all $i \in \N_{0}$.
%
%\eqref{eq:NewtonStepMPE2} can be separated into two parts as
%\begin{equation}\label{eq:NewtonStepMPE}
%\begin{cases}
%P'_{X_{i}}(H_{i}) = -P(X_{i}),\\
%X_{i+1}=X_{i}+H_{i},\end{cases}\quad i=1,2,\ldots.
%\end{equation}
%
%The general approach for solving \eqref{eq:NewtonStepMPE}
%is to solve the $m^{2}\times m^{2}$ linear system derived by Lemma \ref{lem:AXBC} such as
%$$
%\P'_{X_{i}}\vec(H_{i}) = \vec(-P(X_{i})),
%$$
%where
%\begin{equation}\label{eq:MderivativeQME}
%\P'_{X} = \sum_{k=1}^{n} \sum_{l=0}^{k-1} (X^{k-l-1})^{T} \otimes A_{k}X^{l}.
%\end{equation}
%
%%Clearly from Definition \ref{def:Mmat}, if $-A_{1}$ is an $M$-matrix, then so is $-I_{m} \otimes A_{1}$.
%
%
%
%
%%\begin{thm}\label{thm:ConvergenceNewtonMPE}
%%Let the MPE \eqref{eq:MPE} with i) and ii) in Assumption \ref{ass:MPE} satisfy \eqref{eq:SuffMPE}.
%%If $-\mathcal{P}'_{S}$ is an $M$-matrix,
%%then for the Newton iteration \eqref{eq:NewtonStepMPE} with $X_{0}=0$,
%%the sequence $\{X_{i}\}$ is well defined, is monotone nondecreasing, and converges $S$
%%where $S$ is the elementwise nonnegative solution of \eqref{eq:MPE}.
%%Furthermore, $-\mathcal{P}'_{X_{i}}$ is a nonsingular $M$-matrix for each $i \in \N_{0}$.
%%\end{thm}
%%
%%\begin{proof}
%%\cite{SeoSeo2016}
%%\end{proof}
%
%
%
%
%\begin{thm}\label{thm:ConvergenceNewtonMPE}
%Suppose that the MPE \eqref{eq:MPE} satisfies Assumption \ref{ass:MPE} and \eqref{eq:SuffMPE}.
%Then, the Newton sequence $\{X_{i}\}$ with $X_{0} = 0$ is well defined, is monotone nondecreasing,
%and converges to the elementwise minimal positive solution $S$.
%Furthermore, $-\P'_{X_{i}}$ is a nonsingular irreducible $M$-matrix for $i \in \N$, and $-\mathcal{P}'_{S}$ is an irreducible $M$-matrix.
%\end{thm}
%
%
%\begin{proof}
%%From \cite[Theorem 3.8]{SeoSeo2016}, it is sufficient to show that
%%\begin{enumerate}[i)]
%%\item there exist $X_{1}, Y_{0} \in \R^{\mbym}$ such that $X_{1} < Y_{0}$ and $P(X_{1}) > 0 \geq P(Y_{0})$;
%%\item $Y < Z$ implies that $\P'_{Z} - \P'_{Y}$ is nonnegative and irreducible for all $Y, Z \in [X_{1}, Y_{0}] := \{X| X_{1} \leq X \leq Y_{0}\}$;
%%\item $-\P'_{X_{1}}$ is a $Z$-matrix;
%%\item $-\P'_{Y_{0}}$ is an $M$-matrix.
%%\end{enumerate}
%%
%%From \cite[Theorem 2.10]{SeoSeoKim2013}, $-A_{1}^{-1}$ is positive. Then, $X_{1} = -A_{1}^{-1}A_{0}$ is positive.
%%Since $A_{k}$'s are irreducible,
%%\begin{align*}
%%P(X_{1})    & = A_{n}X_{1}^{n} + \ldots + A_{1}X_{1} + A_{0} \\
%%            & = A_{n}X_{1}^{n} + \ldots + A_{2}X_{1}^{2} + A_{1}(-A_{1}^{-1}A_{0}) + A_{0} \\
%%            & = A_{n}X_{1}^{n} + \ldots + A_{2}X_{1}^{2}
%%\end{align*}
%%is also positive.
%%By Theorem \ref{thm:SuffMPE}, there exists the minimal nonnegative solution $S$ of \eqref{eq:MPE}.
%%Moreover, we obtain that $S$ is positive and $X_{1} < S$ from the proof of Theorem \ref{thm:limsupeqrho}.
%%If we put $Y_{0} = S$, then i) holds.
%%
%%Suppose that $Y, Z \in [X_{1}, S]$ and $Y < Z$.
%%Then,
%%\begin{align*}
%%\P'_{Z} - \P'_{Y} &= \sum_{k=1}^{n} \sum_{l=0}^{k-1} \left[ (Z^{k-l-1})^{T} \otimes A_{k}Z^{l} - (Y^{k-l-1})^{T} \otimes A_{k}Y^{l} \right] \\
%%                &\geq \sum_{k=1}^{n} \sum_{l=0}^{k-1} \left[ (Y^{k-l-1})^{T} \otimes A_{k}Z^{l} - (Y^{k-l-1})^{T} \otimes A_{k}Y^{l} \right] \\
%%                &= \sum_{k=1}^{n} \sum_{l=0}^{k-1} (Y^{k-l-1})^{T} \otimes A_{k}\left( Z^{l} - Y^{l} \right) \\
%%                &= \sum_{k=2}^{n} \sum_{l=1}^{k-1} (Y^{k-l-1})^{T} \otimes A_{k}\left( Z^{l} - Y^{l} \right).
%%\end{align*}
%%Since $Y$, $Z^{l} - Y^{l}$ are positive for $l \geq 1$, and $A_{k}$'s are nonnegative irreducible for $k \geq 2$,
%%we obtain that $\P'_{Z} - \P'_{Y} > 0$.
%%Hence, ii) holds.
%%
%%From the following equality
%%\begin{align*}
%%-\P'_{X_{1}} & = -\sum_{k=1}^{n}\sum_{l=0}^{k-1}(X_{1}^{k-l-1})^{T} \otimes A_{k}X^{l} \\
%%            & = -\sum_{k=2}^{n}\sum_{l=0}^{k-1}(X_{1}^{k-l-1})^{T} \otimes A_{k}X^{l} + I_{m} \otimes (-A_{1}),
%%\end{align*}
%%all of off-diagonal entries of $-\P'_{X_{1}}$ are nonpositive, i.e., $-\P'_{X_{1}}$ is a $Z$-matrix.
%%
%%To prove iv), consider the matrix function $G$ defined in \eqref{eq:MPEfix}.
%%Since the convergence rates of any fixed point iterations are at most 1,
%%$\rho(\G'_{S}) \leq 1$ by Theorem \ref{thm:limsupeqrho}.
%%From Lemma \ref{lem:MtimesM} and the following equality,
%%\begin{align*}
%%(I_{m} \otimes -A_{1})(I_{m^{2}} - \G'_{S}) & = (I_{m} \otimes -A_{1})\left(I_{m^{2}} - \sum_{k=2}^{n}\sum_{l=0}^{k-1}(S^{k-l-1})^{T} \otimes (-A_{1}^{-1}A_{k}S^{l})\right)\\
%%                                            & = (I_{m} \otimes -A_{1}) - \sum_{k=2}^{n}\sum_{l=0}^{k-1}(S^{k-l-1})^{T} \otimes A_{k}S^{l} \\
%%                                            & = - \sum_{k=1}^{n}\sum_{l=0}^{k-1}(S^{k-l-1})^{T} \otimes A_{k}S^{l} \\
%%                                            & = -\P'_{S},
%%\end{align*}
%%$-\P'_{S}$ is an $M$-matrix.
%%Therefore, the Newton sequence $\{X_{i}\}$ increases monotonically and converges to the minimal positive solution $S$.
%
%According to the proof of \cite[Theorem 2.1]{Meng2017},
%the elementwise minimal nonnegative solution $S$ of \eqref{eq:MPE} is the limit of the monotone nondecreasing sequence $\{X_{i}^{G}\}_{i=0}^{\infty}$
%which is defined by
%\begin{equation*}
%\begin{cases} X_{i+1}^{G} = G(X_{i}^{G}),\\~\\X_{0}^{G} = 0,\end{cases}
%\end{equation*}
%where
%\begin{equation*}\label{eq:MPEfix}
%G(X) = -A_{1}^{-1}\left( \sum_{k=2}^{n}A_{k}X^{k} + A_{0} \right).
%\end{equation*}
%
%Since $X_{1}^{G} = -A_{1}^{-1}A_{0} > 0$, the solution $S$ is also positive.
%Thus, $S \in \{Y \in \R^{\mbym} | Y>0,~~ F(Y) \leq 0\}$.
%From \cite[Theorem 2.9]{SeoKim2014}, the Newton sequence $\{X_{i}\}$ with $X_{0} = 0$ is well-defined, monotone nondecreasing,
%and converges to the elementwise minimal positive solution $S$.
%Moreover, 
%\begin{equation*}
%-\P'_{X_{i}} = -\sum_{k=1}^{n} \sum_{l=0}^{k-1} (X_{i}^{k-l-1})^{T} \otimes A_{k}X_{i}^{l}
%\end{equation*}
%is a nonsingular $M$-matrix for each $i \in \N_{0}$.
%
%Now, it is sufficient to show that $-\P'_{X_{i}}$ is irreducible for $i \in \N$.
%Since $X_{1} = -A_{1}^{-1}A_{0} > 0$ and $A_{p} \geq 0$ for all $p \geq 2$,
%\begin{equation*}
%\sum_{k=2}^{n} \sum_{l=0}^{k-1} (X_{i}^{k-l-1})^{T} \otimes A_{k}X_{i}^{l}
%\geq \sum_{k=2}^{n} (X_{i}^{k-1})^{T} \otimes A_{k} \geq 0.
%\end{equation*}
%From that $X_{i} > 0$, we obtain that $\sum_{k=2}^{n} (X_{i}^{k-1})^{T} \otimes A_{k}$ is irreducible
%if and only if $\sum_{k=2}^{n} \ones{\mbym} \otimes A_{k} = \ones{\mbym} \otimes \left(\sum_{k=2}^{n}A_{k}\right)$ is irreducible.
%Since $\sum_{k=2}^{n}A_{k}$ is irreducible,
%\begin{equation*}
%\sum_{k=2}^{n} \sum_{l=0}^{k-1} (X_{i}^{k-l-1})^{T} \otimes A_{k}X_{i}^{l}
%\end{equation*}
%is a irreducible nonnegative matrix.
%Therefore,
%\begin{equation*}
%-\P'_{X_{i}} = -I_{m} \otimes A_{1} -\sum_{k=2}^{n} \sum_{l=0}^{k-1} (X_{i}^{k-l-1})^{T} \otimes A_{k}X_{i}^{l}
%\end{equation*}
%is also irreducible.
%
%\end{proof}
%
%
%The next theorem follows directly from Theorem \ref{thm:ConvergenceNewtonMPE} 
%and the well known local quadratic convergence of Newton's method.
%
%\begin{thm}\label{thm:ConvRateQuadratic}%{\rm (cf.\cite[Theorem 4.1.9]{kim00})}
%If the matrix $-\mathcal{P}'_{S}$ in Theorem \ref{thm:ConvergenceNewtonMPE}
%is nonsingular, then for $X_{0}=0$, the Newton sequence
%$\{X_{i}\}$ converges to $S$, quadratically.
%\end{thm}
%
%
%%\begin{proof}
%%Let $\delta \in (0,1]$ be given sufficiently small.
%%By the hypothesis, the Fr$\mathrm{\acute{e}}$chet derivative $P'_{S}$ is an invertible map.
%%So,
%%\begin{equation}
%%c_{0} = \inf\{ \|P'_{X}(H)\| : \|H\|=1, \|X - S\| \leq \delta \} > 0.
%%\end{equation}
%%Since the sequence $\{\|X_{i} - S\|\}$ is monotone nonincreasing,
%%it is bounded by $\|S\|$.
%%Hence, there exists $a_{0} > 0$ such that
%%\begin{equation}\label{eq:bigOleq}
%%\left\|O(\|X_{i} - S\|^{2})\right\| \leq a_{0}\|X_{i} - S\|^{2}.
%%\end{equation}
%%With \eqref{eq:Taylor} and \eqref{eq:bigOleq}, we obtain the following inequality,
%%\begin{align*}
%%\|X_{i+1} - S\| &= \|X_{i} - S - P_{X_{i}}^{\prime -1}(P(X_{i}))\| \\
%%                &= \|(X_{i} - S) - P_{X_{i}}^{\prime -1}(P(X_{i}) - P(S))\| \\
%%                &\leq \|P_{X_{i}}^{\prime -1}\| \| P'_{X_{i}}(X_{i} - S) - (P(X_{i}) - P(S)) \| \numberthis \label{eq:QuadConvIneq}\\
%%                &\leq (1/c_{0})\left\|O(\|X_{i} - S\|^{2})\right\| \\
%%                &\leq (a_{0}/c_{0})\|X_{i} - S\|^{2}.
%%\end{align*}
%%Therefore, the sequence $\{X_{i}\}$ converges to $S$, quadratically.
%%\end{proof}
%
%
%
%\section{Analysis for the Singular $M$-matrix $-\P'_S$}\label{sec:singular}
%According to Theorem \ref{thm:ConvRateQuadratic}, the Newton iteration \eqref{eq:NewtonStepMPE2} converges quadratically
%if $-\P'_{S}$ is nonsingular.
%In this section, we will see the convergence rate of \eqref{eq:NewtonStepMPE2} when $-\P'_{S}$ is a singular $M$-matrix.
%%we will see the Newton sequence also converges to the solution but linearly.
%If $P'_{S}$ is not invertible, then $P'_{S}$ has a null space $\mathcal{N}=\text{Ker}(P'_{S})$
%and closed range $\mathcal{M}=\text{Im}(P'_{S})$.
%Suppose that the direct sum $\mathcal{N} \oplus \mathcal{M} = \mathbb{R}^{\mbym}$.
%Then, we can define $\pjn$ to be the projection onto $\mathcal{N}$
%parallel to $\mathcal{M}$ and $\pjm=I-\pjn$.
%For a nonzero matrix $N_{0}\in\mathcal{N}$, define the map $\mathcal{B}_{N_{0}}:\mathcal{N}\rightarrow\mathcal{N}$
%given by
%\begin{equation}\label{eq:secondDerivativeProjection}
%\mathcal{B}_{N_{0}}(N)=\pjn P_{S}^{\prime \prime}(N_{0},N).
%\end{equation}
%In fact, $\mathcal{B}_{N_{0}}$ is a linear map.
%The main result of this section is an application of the following theorem 
%which shows the local convergence and the convergence rate of Newton's method under some conditions.
%
%
%
%\begin{thm}{\rm (cf. \cite[Theorem 1.5]{Guo1998M}, \cite[Theorem 1.1]{Kelley1986})}\label{thm:ApplicationTheorem}
%Let $\mathcal{B}_{N_{0}}$ in \eqref{eq:secondDerivativeProjection} be invertible for some nonzero $N_{0}\in\mathcal{N}$,
%$\mathcal{N} = \mathrm{span}\{N_{0}\}\oplus\mathcal{N}_{1}$ for some subspace $\mathcal{N}_{1}$,
%and let
%\begin{equation}\label{eq:Wspace}
%W(\rho, \theta, \eta) =
%\left\{ X \left|\,\begin{array}{l}0 < \|X - S\| < \rho, \|\pjm(X - S)\| \leq \theta \|\pjn(X - S)\|,\\ 
%				\|(\pjn-\mathbf{P}_{0})(X - S)\| \leq \eta \|\pjn(X - S)\|\end{array}\right.\right\},
%\end{equation}
%where $\mathbf{P}_{0}$ is the projection onto $\mathrm{span}\{N_{0}\}$
%parallel to $\mathcal{N}_{1}\oplus\mathcal{M}$. If $X_{0}\in W(\rho_{0},\theta_{0},\eta_{0})$
%for $\rho_{0},\theta_{0},\eta_{0}$ sufficiently small, then the Newton
%sequence $\{X_{i}\}$ is well defined and $\| P_{X_{i}}^{\prime -1}\|\leq c\|X_{i} - S\|^{-1}$
%for all $i\geq1$ and some constant $c>0$. Moreover,
%$$
%\lim_{i \rightarrow \infty}\frac{\|X_{i+1} - S\|}{\|X_{i} - S\|} = \frac{1}{2},
%\qquad \lim_{i \rightarrow \infty}\frac{\|\pjm(X_{i} - S)\|}{\|\pjn(X_{i} - S)\|^{2}}=0.
%$$
%\end{thm}
%
%
%To analyze convergence of Newton's method when $-\mathcal{P}'_{S}$ is singular,
%we will show that \eqref{eq:MPE} satisfies the conditions of Theorem \ref{thm:ApplicationTheorem}.
%From this point, for convenience, we let $\tXi = X_{i} - S$.
%%Before proving the following lemma, we use $\text{unvec}$ operator from $\mathbb{R}^{n^{2}}$ onto $\R^{\nbyn}$
%%which is the inverse of the vec operator.
%
%
%
%
%\begin{lemma}\label{lem:BisInvertible}
%Suppose the matrix polynomial equation \eqref{eq:MPE} satisfies Assumption \ref{ass:MPE}.
%If the matrix $-\mathcal{P}'_{S}$ is a singular $M$-matrix,
%then $0$ is a simple eigenvalue of $-\mathcal{P}'_{S}$, $\mathcal{N}\oplus\mathcal{M}=\mathbb{R}^{m \times m}$,
%$\mathcal{N}$ is one-dimensional and the map $\mathcal{B}_{N_{0}}$ is invertible
%for some nonzero $N_{0}\in\mathcal{N}$.
%\end{lemma}
%
%
%\begin{proof}
%%From \eqref{eq:Mderivative2}, $-\mathcal{D}_{S}=rI_{n^2}-\mathbf{N}(S)$
%%where $\mathbf{N}(S)=I_{n}\otimes T_{0}+S^{T}\otimes A+I_{n}\otimes AS$.
%Since $S$ is positive and $A_{k}$'s are irreducible, $-\P'_{S}$ is irreducible.
%%Hence, $\mathbf{N}(S)$ is also irreducible.
%Then, by Perron-Frobenius Theorem \cite[Theorem 8.4.4]{Horn1985},
%$0$ is a simple eigenvalue of $\P'_{S}$ with a positive eigenvector. Thus, we can find $n^{2}$ linearly independent
%vectors $\chi_{1},\chi_{2},\cdots \chi_{n^{2}}$ such that $\chi_{1}>0$ and
%
%\begin{equation}\label{eq:invUDsU}
%\X^{-1}\P'_{S}\X=\bmatrix{
%0 & 0\\
%0 & \mathcal{D}}, ~\textrm{where }\X=\left[\begin{array}{c|c|c|c} \phantom{1}&\phantom{1}&\phantom{1}&\phantom{1}\\ \chi_{1}&\chi_{2}&\cdots&\chi_{n^{2}}\\\phantom{1}&\phantom{1}&\phantom{1}&\phantom{1}
%\end{array} \right]
%%X^{-1}\mathcal{D}_{S}X=\bmatrix{
%%0 & 0\\
%%0 & \mathcal{D}_{22}}, ~\textrm{where }X=\left[\textrm{\begin{tabular}{c|c|c|c} \multirow{2}{*}{$x_{1}$}&\multirow{2}{*}{$x_{2}$}&\multirow{2}{*}{$\cdots$}&\multirow{2}{*}{$x_{n^{2}}$}\\
%%& & &
%%\end{tabular}} \right]
%\end{equation}
%and $\mathcal{D}$
%is an $(n^{2}-1)\times(n^{2}-1)$ nonsingular matrix.
%By the same way, we also have a positive vector $\psi$ such that $\psi^{T}\P'_{S} = 0$.
%Now, $P'_{S}(N) = 0$ if and only if $\P'_{S}\mathrm{vec}(N)=0$.
%From \eqref{eq:invUDsU}, $\P'_{S}\mathrm{vec}(N) = 0$ if and only if $\mathrm{vec}(N) \in \mathrm{span}(\chi_{1})$,
%in which case we write $N = a\vec^{-1}(\chi_{1})$ for some nonzero $a \in \R$.
%Thus, $\mathcal{N} = \mathrm{span}(\vec^{-1}(\chi_{1}))$.
%Simiarly, $\mathcal{M} = \mathrm{span}(\vec^{-1}(\chi_{2}), \ldots, \vec^{-1}(\chi_{n^{2}}))$.
%Therefore, $\mathcal{N}$ is one-dimensional and $\mathbb{R}^{m \times m}=\mathcal{N}\oplus\mathcal{M}$.
%
%To prove the map $\mathcal{B}_{N_{0}}$ is invertible for a nonzero matrix $N_{0} \in \mathcal{N}$ , we only need to show that
%$$
%\pjn\left( P''_{S}(N_{0},N) \right) \neq 0,
%$$
%for all nonzero $N \in \mathcal{N}$ because $\mathcal{B}_{N_{0}}$ is linear and $\mathcal{N}$ is one-dimensional.
%Since $\vec^{-1}(\chi_{1})>0$ and $S > 0$, we have
%\begin{align*}
%P''_{S}(N_{0},N) & = \sum_{k=2}^{n} \sum_{l=0}^{k-2} \sum_{j=0}^{l} A_{k}\left( S^{l}N_{0}S^{j}NS^{n-l-j-2} + S^{l}NS^{j}N_{0}S^{n-l-j-2} \right) \\
%                 & = 2ab \sum_{k=2}^{n} \sum_{l=0}^{k-2} \sum_{j=0}^{l} A_{k}\left( S^{l} \vec^{-1}(\chi_{1}) S^{j} \vec^{-1}(\chi_{1}) S^{n-l-j-2} \right) \neq 0
%\end{align*}
%where $N = a\vec^{-1}(\chi_{1})$ and $N_{0} = b\vec^{-1}(\chi_{1})$.
%Moreover, $P''_{S}(N_{0},N)$ is either positive or negative.
%
%On the other hand,
%$$
%\vec\left( P''_{S}(N_{0},N) \right) = k_{1}\chi_{1}+k_{2}\chi_{2}+\cdots+k_{n^{2}}\chi_{n^{2}}
%$$
%for some real numbers $k_{1},k_{2},\cdots,k_{n^{2}}$.
%By Fundamental theorem of linear algebra in \cite{strang2006linear} and Lemma 6.3.10 in \cite{Horn1985}, we have
%\begin{equation}\label{eq:yTPtYx}
%\psi^{T} \vec\left( P''_{S}(N_{0},N) \right) = k_{1}\psi^{T}\chi_{1}.
%\end{equation}
%Since $P''_{S}(N_{0},N)$ is either positive or negative and $\psi$ is positive, the left side of \eqref{eq:yTPtYx} is also either positive or negative.
%So, $k_{1}$ cannot be zero.
%Therefore,
%$$
%\pjn \left( P''_{S}(N_{0},N) \right) = k_{1}\vec^{-1}(\chi_{1}) \neq 0.
%$$
%
%\end{proof}
%
%
%
%
%
%\begin{lem}\label{lem:TaylorLemma}
%	Let $S$ be the minimal positive solution of \eqref{eq:MPE} with Assumption \ref{ass:MPE},
%	and let $\{X_{i}\}_{i=0}^{\infty}$ be a Newton sequence in \eqref{eq:NewtonStepMPE2}. Then,
%	$$
%	\| P({X}_{i})\|\leq a\|\tXi\|^{2}+b\|\tXi\|\|\tilde{X}_{i-1}\|+c\|\tilde{X}_{i-1}\|^{2}
%	$$
%	for some positive real numbers $a,b,c$.
%\end{lem}
%
%\begin{proof}
%	From Taylor's Theorem and putting $S = X_{i-1} - \tilde{X}_{i-1}$, we have
%	\begin{equation}\label{eq:Taylor}
%	P(X_{i}) = P(S) + P'_{S}(X_{i} - S) + O(\|X_{i} - S\|^{2}),
%	\end{equation}
%	and
%	\begin{equation*}
%	0 = P(S) = P(X_{i-1} - \tilde{X}_{i-1}) = P(X_{i-1}) - P'_{X_{i-1}}(\tilde{X}_{i-1}) + O(\|\tilde{X}_{i-1}\|^{2}),
%	\end{equation*}
%	which is equivalent to
%	\begin{equation}\label{eq:TaylorMPE3}
%	- P(X_{i-1}) + P'_{X_{i-1}}(\tilde{X}_{i-1}) = O(\|\tilde{X}_{i-1}\|^{2}).
%	\end{equation}
%	From \eqref{eq:NewtonStepMPE}, we have
%	\begin{equation}\label{eq:NewtonStepAnother}
%	0 = P'_{X_{i-1}}(X_{i} - X_{i-1}) + P(X_{i-1}),
%	\end{equation}
%	and clearly
%	\begin{equation}
%	X_{i} - X_{i-1} = \tXi - \tilde{X}_{i-1}.
%	\end{equation}
%	If we subtract \eqref{eq:NewtonStepAnother} from \eqref{eq:Taylor} 
%	and substitute \eqref{eq:TaylorMPE3}, we obtain
%	\begin{align*}
%	P(X_{i}) & = P(S) + P'_{S}(\tXi) - P(X_{i-1}) - P'_{X_{i-1}}(\tXi - \tilde{X}_{i-1}) + O(\|\tXi\|^{2}) \\
%	& = P'_{S}(\tXi) - P'_{X_{i-1}}(\tXi) + O(\| \tXi \|^{2}) + O(\|\tilde{X}_{i-1}\|^{2}).
%	\end{align*}
%	Putting $S = X_{i-1} - \tilde{X}_{i-1}$ in the previous equality,
%	\begin{align*}
%	P(X_{i}) & = \sum_{k=1}^{n}\sum_{l=0}^{k-1}A_{k}(X_{i-1} - \tilde{X}_{i-1})^{l}\tXi(X_{i-1} - \tilde{X}_{i-1})^{k-l-1} \\
%	& \qquad\qquad\qquad\qquad - P'_{X_{i-1}}(\tXi)	+ O(\| \tXi \|^{2}) + O(\|\tilde{X}_{i-1}\|^{2})\\
%	& = \sum_{k=1}^{n}\sum_{l=0}^{k-1}A_{k}X_{i-1}^{l}\tXi X_{i-1}^{k-l-1} + O(\|\tXi\|\|\tilde{X}_{i-1}\|)\\
%	& \qquad\qquad\qquad\qquad - P'_{X_{i-1}}(\tXi)	+ O(\| \tXi \|^{2}) + O(\|\tilde{X}_{i-1}\|^{2})\\
%	& = P'_{X_{i-1}}(\tXi)+O(\|\tXi\|\|\tilde{X}_{i-1}\|)- P'_{X_{i-1}}(\tXi)+O(\| \tXi \|^{2}) + O(\|\tilde{X}_{i-1}\|^{2})\\
%	& = O(\| \tilde{X}_{i-1} \|^{2}) + O(\| \tXi \|\| \tilde{X}_{i-1} \|) + O(\| \tXi \|^{2}).
%	\end{align*}
%	Since $\|\cdot\|$ is a multiplicative matrix norm on $\mathbb{R}^{m\times m}$,
%	we have required result.
%\end{proof}
%
%
%
%
%\begin{lem}\label{lem:DominatedPMconvergence}
%For any fixed $\theta > 0$, let
%$$
%\mathcal{Q} = \{i \in \N \cup \{0\}|\|\pjm(\tXi)\| > \theta \|\pjn(\tXi)\|\}
%$$
%where $\{X_{i}\}$ is the Newton sequence in Theorem \ref{thm:ConvergenceNewtonMPE}.
%Then, there exist an integer $i_{0}$ and a constant $c > 0$ such that
%$\| \tXi\| \leq c \| \tilde{X}_{i-1} \|^{2}$ for all $i\geq i_{0}$ in $\mathcal{Q}$.
%\end{lem}
%
%\begin{proof}
%Using Taylor's Theorem and the fact that $P'_{S}\left(\pjn(\tXi)\right) = 0$,
%\begin{equation}\label{eq:TaylorMPE}
%P(X_{i}) = P(S) + P'_{S}(\tXi) + O(\|\tXi\|^{2}) = P'_{S}\left(\pjm(\tXi)\right) + O(\|\tXi\|^{2}).
%\end{equation}
%Since $P'_{S}|_{\mathcal{M}} : \mathcal{M} \rightarrow \mathcal{M}$ is
%invertible, $\left\| P'_{S}\left(\pjm(\tXi)\right)\right\| \geq c_{1}\|\pjm(\tXi)\|$
%for some constant $c_{1}>0$.
%For $i\in\mathcal{Q}$, we have
%\begin{equation*}
%\|\tXi\| \leq \left\| \pjm(\tXi)\right\| + \left\| \pjn(\tXi)\right\|
%\leq (\theta^{-1} + 1) \left\| \pjm(\tXi)\right\|.
%\end{equation*}
%Thus by \eqref{eq:TaylorMPE},
%\begin{equation}\label{eq:NormofhatXi}
%\|P(X_{i})\| \geq c_{1}\|\pjm(\tXi)\| - c_{2}\|\tXi\|^{2} \geq c_{1}(\theta^{-1}+1)^{-1}\|\tXi\| - c_{2}\|\tXi\|^{2}.
%\end{equation}
%On the other hand, from Lemma \ref{lem:TaylorLemma}, we have
%$$
%\| P({X}_{i})\|\leq c_{3}\|\tXi\|^{2}+c_{4}\|\tilde{X}_{i-1}\|\|\tXi\|+c_{5}\|\tilde{X}_{i-1}\|^{2}.
%$$
%From \eqref{eq:NormofhatXi} and the fact that $X_{i}\neq S$ for
%any $i$, we have
%$$
%c_{1}(\theta^{-1}+1)^{-1} - c_{2}\|\tXi\| \leq c_{3}\|\tXi\| + c_{4}\|\tilde{X}_{i-1}\| + c_{5}\frac{\|\tilde{X}_{i-1}\|^{2}}{\|\tXi\|}.
%$$
%Since $\tXi$ converges to $0$ by Theorem \ref{thm:ConvergenceNewtonMPE},
%we can find an $i_{0}$ such that $\|\tXi\|\leq c\|\tilde{X}_{i-1}\|^{2}$
%for all $i\geq i_{0}$.\end{proof}
%\begin{cor}
%Assume that, for given $\theta>0$, $\|\pjm(\tXi)\|>\theta\|\pjn(\tXi)\|$
%for all $i$ large enough. Then $X_{i}\rightarrow S$ quadratically.
%\end{cor}
%
%
%When $P'_{S}$ is singular practically the Newton sequence
%converges linearly, according to the corollary we conclude that \textit{the
%error will generally be dominated by its $\mathcal{N}$ component}\cite{Guo1998M}.
%From Lemmas \ref{lem:BisInvertible} and \ref{lem:DominatedPMconvergence}
%we have the following main theorem.
%
%\begin{thm}\label{thm:LinearConvergence1over2}
%If $-\P'_{S}$ is a singular $M$-matrix and the convergence rate
%of the Newton sequence $\{X_{i}\}$ in Theorem \ref{thm:ConvergenceNewtonMPE}
%is not quadratic, then $\| P_{X_{i}}^{\prime -1}\|\leq c\| \tXi\|^{-1}$
%for all $i \geq 1$ and some constant $c > 0$. Moreover,
%$$
%\lim_{i \rightarrow \infty} \frac{\|\tXii\|}{\|\tXi\|} = \frac{1}{2},
%\qquad \lim_{i\rightarrow\infty} \frac{\|\pjm(\tXi)\|}{\|\pjn(\tXi)\|^{2}}=0.
%$$
%
%\end{thm}
%
%
%\section{A Modified Newton Method}\label{sec:modified}
%Under the conditions in Theorem \ref{thm:LinearConvergence1over2}, the convergence rate of the Newton sequence is $1/2$.
%Furthermore, $\|\pjm(\tXi)\| / \|\pjn(\tXi)\|^{2}$ converges to $0$,
%i.e.,$\|\pjm(\tilde{X}_{i_{0}})\| < \e\|\pjn(\tilde{X}_{i_{0}})\|$ holds for sufficiently small $\e > 0$ 
%and large integer $i_{0}$ to make $\|\pjn(\tilde{X}_{i_{0}})\| < 1$.
%Intuitivley, we understand that $\pjm(\tilde{X}_{i_{0}})$ is almost terminated,
%and $\{\tXi\}_{i=i_{0}}^{\infty}$ is located near a one-dimensional subspace $\mathcal{N}$.
%Then, we will give a modified Newton method which has faster convergence than the pure one.
%
%\begin{lem}\label{lem:PjnMonotone}
%Let $\{X_{i}\}$ be the Newton sequence in Theorem \ref{thm:ConvergenceNewtonMPE},
%and let the derivative $P'_{S}$ be singular.
%Suppose that there exists $i_{0} \in \N_{0}$ such that $i \geq i_{0}$ implies that
%\begin{equation}
%\|\pjm(\tXi)\| < \e\|\pjn(\tXi)\|,
%\end{equation}
%for $0 < \e \leq 1$.
%Then, $\pjn(\tXi)$ is negative for $i \geq i_{0}$.
%\end{lem}
%
%\begin{proof}
%From the proof of Lemma \ref{lem:BisInvertible}, $V_{1} := \vec^{-1}(\chi_{1})$ is a positive basis for $\mathcal{N}$.
%Let $c_{i,1}$ be a scalar such that $\pjn(\tXi) = c_{i,1}V_{1}$.
%Suppose that $c_{i,1} \geq 0$. Then, $\pjn(\tXi) \geq 0$.
%Since $\pjn(\tXi) + \pjm(\tXi) = \tXi \leq 0$, we obtain that $0 \leq \pjn(\tXi) \leq -\pjm(\tXi)$.
%Thus,
%\begin{equation}
%\|\pjn(\tXi)\| \leq \|\pjm(\tXi)\| < \e\|\pjn(\tXi)\|.
%\end{equation}
%It means that $\e > 1$, i.e., it contradicts the hypothesis.
%Therefore, $c_{i,1} < 0$ and $\pjn(\tXi) < 0$.
%\end{proof}
%
%Consider a polynomial $f(x) = px^{3} + 2px^{2} + (9p+1)x - 1$ for $p > 0$.
%Since $f(0) = -1$ and $f(1) = 12p$, $f$ has a root in the interval $(0,1)$.
%From that $f'(x) = 3px^{2} + 4px + 9p + 1 > 0$ for all $x > 0$,
%$f$ is monotone increasing in $(0,1)$, i.e., the root $t$ of $f$ in $(0,1)$ is unique.
%Hence, for $x \in (0,t)$, it holds that
%\begin{equation}
%\dfrac{9x + 2x^{2} + x^{3}}{1 - x} < \dfrac{1}{p}.
%\end{equation}
%The previous inequality is useful to prove the following theorem.
%
%\begin{thm}\label{thm:modifiedNewton}
%Let $Y_{i+1} = X_{i} - 2P_{X_{i}}^{\prime -1}(P(X_{i}))$, $p = \|\pjn\|$, and let $\e \in (0,t)$ be given where $t$ is the real root of
%$f(x) = px^3 + 2px^2 + (9p + 1)x - 1$ in $(0,1)$.
%Suppose that $i \geq i_{0}$ implies that
%\begin{equation}
%\left| \dfrac{\|\tXi\|}{\|\tXii\|} - 2 \right| < \e,\qquad \|\pjm(\tXi)\| < \e\|\pjn(\tXi)\|
%\end{equation}
%for some $i_{0} \in \N_{0}$. Then, for $i \geq i_{0}$,
%\begin{equation}
%\|Y_{i+1} - S\| < \|\tXii\|.
%\end{equation}
%\end{thm}
%
%\begin{proof}
%From the definition of $Y_{i+1}$, we get that
%\begin{align*}
%Y_{i+1} - S &= X_{i} - 2P_{X_{i}}^{\prime -1}(P(X_{i})) - S \\
%            &= 2\left(X_{i} - P_{X_{i}}^{\prime -1}(P(X_{i}))\right) - X_{i} - S \\
%            &= 2X_{i+1} - X_{i} - 2S + S \\
%            &= 2\tXii - \tXi.
%\end{align*}
%So, we will show that $\|2\tXii - \tXi\| < \|\tXii\|$.
%From the hypothesis, we obtain that
%\begin{equation}
%(1-\e)\|\pjn(\tXi)\| < \|\tXi\| < (1+\e)\|\pjn(\tXi)\|.
%\end{equation}
%It yields two following inequalities,
%\begin{equation}
%\begin{array}{c}
%2-\e < \dfrac{\|\tXi\|}{\|\tXii\|} < \dfrac{(1+\e)\|\pjn(\tXi)\|}{(1-\e)\|\pjn(\tXii)\|},\\~\\
%\dfrac{(1-\e)\|\pjn(\tXi)\|}{(1+\e)\|\pjn(\tXii)\|} < \dfrac{\|\tXi\|}{\|\tXii\|} < 2+\e.
%\end{array}
%\end{equation}
%From the previous two inequalities, we get the inequality
%\begin{equation}
%\dfrac{(2-\e)(1-\e)}{1+\e} < \dfrac{\|\pjn(\tXi)\|}{\|\pjn(\tXii)\|} = \dfrac{c_{i,1}}{c_{i+1,1}} < \dfrac{(2+\e)(1+\e)}{1-\e},
%\end{equation}
%where $c_{i,1}$ and $c_{i+1,1}$ are scalars which are in the proof of Lemma \ref{lem:PjnMonotone}, i.e., they are negative.
%It is obtained that
%\begin{equation}
%\dfrac{-5\e + \e^2}{1+\e}(-c_{i+1,1}) < 2c_{i+1,1} - c_{i,1} < \dfrac{5\e + \e^2}{1-\e}(-c_{i+1,1}).
%\end{equation}
%Since $\dfrac{5\e + \e^2}{1-\e} - \dfrac{5\e - \e^2}{1+\e}$ is positive for $0 < \e < 1$,
%\begin{equation}
%|2c_{i+1,1} - c_{i,1}| < \dfrac{5\e + \e^2}{1-\e}|c_{i+1,1}|.
%\end{equation}
%
%Therefore, we get the following inequality,
%\begin{align*}
%\|2\tXii - \tXi\| &= \|2\pjn(\tXii) - \pjn(\tXi) + 2\pjm(\tXii) - \pjm(\tXi)\| \\
%                  &\leq \|2\pjn(\tXii) - \pjn(\tXi)\| + \|2\pjm(\tXii)\| + \|\pjm(\tXi)\| \\
%                  &< \|2\pjn(\tXii) - \pjn(\tXi)\| + 2\e\|\pjn(\tXii)\| + \e\|\pjn(\tXi)\| \\
%                  &= |2c_{i+1,1} - c_{i,1}|\|V_{1}\| + 2\e\|\pjn(\tXii)\| + \e\|\pjn(\tXi)\| \numberthis \label{eq:YiiLessXii}\\
%                  &< \dfrac{5\e + \e^2}{1-\e}|c_{i+1,1}|\|V_{1}\| + 2\e\|\pjn(\tXii)\| + \dfrac{(2+\e)(1+\e)}{1-\e}\e\|\pjn(\tXii)\| \\
%                  &= \dfrac{5\e + \e^2}{1-\e}\|\pjn(\tXii)\| + 2\e\|\pjn(\tXii)\| + \dfrac{(2+\e)(1+\e)}{1-\e}\e\|\pjn(\tXii)\| \\
%                  &= \dfrac{9\e + 2\e^2 + \e^3}{1-\e}\|\pjn(\tXii)\| \\
%                  &\leq \dfrac{9\e + 2\e^2 + \e^3}{1-\e}p\|\tXii\|.
%\end{align*}
%Since $\e \in (0,t)$, $\dfrac{9\e + 2\e^2 + \e^3}{1-\e} < \dfrac{1}{p}$ holds,
%\begin{equation}
%\|Y_{i+1} - S\| = \|2\tXii - \tXi\| < \|\tXii\|.
%\end{equation}
%\end{proof}
%
%%If we already know the solution for the given equation,
%%we can choose $Y_{i+1}$ as the next step of a suitable $X_{i}$ after checking the conditions of Theorem \ref{thm:modifiedNewton}.
%%But, we don't know the solution usually.
%%Hence, choosing $Y_{i+1}$ is dangerous for some cases, specially, the cases that the Fr\'{e}chet derivative at the solution is nonsingular.
%%We need to refer the drawing back algorithm which is introduced in \cite{Guo1998M} for algebraic Riccati equations.
%The theoretical result in Theorem \ref{thm:modifiedNewton} suggests a modified Newton method, as in \cite{Guo1998M} for algebraic Riccati equations.
%The main ideas of the algorithm are that we choose $X_{i+1}$ as the next step of $X_{i}$ if $\|P(Y_{i+1})\| \geq \eta$ for given tolerance $\eta$
%and the iteration is terminated if $\|P(X_{i+1})\| < \eta$ or $\|P(Y_{i+1})\| < \eta$.
%
%
%\begin{algo}\label{algo:modifiedNewton}
%The Modified Newton Method for the Matrix Polynomial Equations for the given tolerance $\eta$.
%
%\begin{enumerate}[\small 1.]
%\item $X_{0} \leftarrow 0$;
%\item Calculate $H$ such that $\P'_{X_{0}}\vec(H) = -\vec(P(X_{0}))$;
%\item $X_{0} \leftarrow X_{0} + 2H$;
%\item If $\|P(X_{0})\| < \eta$, then go to step 7;
%\item $X_{0} \leftarrow X_{0} - H$;
%\item If $\|P(X_{0})\| \geq \eta$, then go to step 2;
%\item $S \leftarrow X_{0}$.
%\end{enumerate}
%\end{algo}
%
%
%\section{Numerical Experiments}\label{sec:experiment}
%
%In this section, we compare the effectiveness of the modified Newton method and the pure one for the MPE.
%The experiments are made with MATLAB R2016a.
%The tolerance of the algorithm of Newton's method is given by $\eta = m \times 10^{-16}$
%and we will stop the iteration if $\|P(X_{i+1})\| < \eta$.
%
%\begin{exam}\label{ex:MPE}
%This example is given to check Theorem \ref{thm:modifiedNewton}, theoretically.
%So, we give the example whose solution is easy to be found,
%and the calculations of the iterations in the experiments are computed up to 100 digits with \texttt{vpa} function.
%Let an MPE \eqref{eq:MPE} with degree $n = 6$ be given with the following coefficients
%\begin{equation}\label{eq:MPEex}
%\left\{\begin{array}{ll}
%A_{k} = a_{k}W,&\textrm{for }k = 0, 2, 3, 4, 5, \\
%A_{1} = a_{1}W - I_{m},&\\
%A_{6} = W,&
%\end{array}\right.
%\end{equation}
%where
%\begin{equation}
%W = \dfrac{1}{6200(m-1)}(\ones{m \times m} - I_{m}),\quad\text{and}\quad
%\left\{\begin{array}{ll}
%a_{0} = 4096, & a_{1} = 56,\\
%a_{2} = 384, & a_{3} = 1312,\\
%a_{4} = 321, & a_{5} = 30.
%\end{array}\right.
%\end{equation}
%Then, $A_{k}$'s satisfy Assumption \ref{ass:MPE} and \eqref{eq:SuffMPE}.
%Hence, the MPE has the minimal positive solution $S$.
%
%Let $m = 3$. Then, the minimal nonnegative solution
%\begin{equation*}
%S = \bmatrix{\dfrac{2r+1}{3} & \dfrac{1-r}{3} & \dfrac{1-r}{3} \\~\\
%			\dfrac{1-r}{3} & \dfrac{2r+1}{3} & \dfrac{1-r}{3} \\~\\
%			\dfrac{1-r}{3} & \dfrac{1-r}{3} & \dfrac{2r+1}{3}},
%\end{equation*}
%where $r \approx -0.3287191$ which is the nearest real root to $0$ of the equation $x^6 + 30x^5 + 321x^4 + 1312x^3 + 384x^2 + 12456x + 4096 = 0$.
%Furthermore, $-\P'_{S}$ is a singular $M$-matrix.
%Therefore, we can calculate $\pjn$ and $\pjm$, easily.
%In this example, in fact, $-\P'_{S}$ is symmetric, so, $\pjn$ and $\pjm$ are orthogonal projections, i.e., $p = \|\pjn\| = 1$.
%Thus, $t \approx 0.097985683$ is the real root of $x^3 + 2x^2 + 10x - 1$.
%
%If we calculate the Newton iteration $\{X_{i}\}_{i=1}^{\infty}$ in \eqref{eq:NewtonStepMPE2} with $X_{0} = 0$
%and $\{Y_{i}\}_{i=1}^{\infty}$ in Theorem \ref{thm:modifiedNewton},
%then we obtain Figures \ref{figure:6degree1}, \ref{figure:6degree2}, \ref{figure:6degree3} as results of 
%$\left| \frac{\|\tXi\|}{\|\tXii\|} - 2 \right|$, $\|\pjn(\tXi)\|$, $\|\pjm(\tXi)\|$, $\|X_{i} - S\|$, and $\|Y_{i} - S\|$.
%Figure \ref{figure:6degree1} shows that $\left| \frac{\|\tXi\|}{\|\tXii\|} - 2 \right| < t$ if $i \geq 2$.
%From Figure \ref{figure:6degree2}, we get that $\|\pjm(\tXi)\| < t\|\pjn(\tXi)\|$ when $i \geq 1$.
%Finally, we see that $\|Y_{i} - S\| < \|X_{i} - S\|$ for $i \geq 2$ through Figure \ref{figure:6degree3}.
%
%This experiment shows that, for given $i$, $Y_{i}$ is closer to the solution than $X_{i}$ if it satisfies the conditions of Theorem \ref{thm:modifiedNewton}.
%Furthermore, we obtain that $Y_{i}$'s for $i \geq 13$ are closer to the solution than any $X_{i}$'s.
%It means that we do not need to compute $X_{i}$ for $i \geq 13$ since $Y_{13}$ is thought to be a ``numerical'' minimal nonnegative solution
%which is close sufficiently to the ``mathematical'' minimal nonnegative solution.
%
%%\begin{figure}[t!]
%%	\begin{center}
%%		\resizebox{10cm}{8cm}{\includegraphics{NM1.pdf}}
%%		\caption{\label{figure:6degree1}Checking which $i$ satisfies $\left| \dfrac{\|\tXi\|}{\|\tXii\|} - 2 \right| < t$ }
%%	\end{center}
%%\end{figure}
%%
%%\begin{figure}[t!]
%%	\begin{center}
%%		\resizebox{10cm}{8cm}{\includegraphics{NM2.pdf}}
%%		\caption{\label{figure:6degree2}Checking which $i$ satisfies $\|\pjm(\tXi)\| < t\|\pjn(\tXi)\|$}
%%	\end{center}
%%\end{figure}
%%
%%\begin{figure}[t!]
%%	\begin{center}
%%		\resizebox{10cm}{8cm}{\includegraphics{NM3.pdf}}
%%		\caption{\label{figure:6degree3}Comparison with $\|X_{i} - S\|$ and $\|Y_{i} - S\|$}
%%	\end{center}
%%\end{figure}
%\end{exam}
%
%\begin{exam}\label{ex:nocommonW}{\rm (cf.\cite[Example 1]{C.He2004})}
%	In this example, we use Algorithm \ref{algo:modifiedNewton} for the following MPEs and $m = 8$.
%	Let MPEs
%	\begin{align}
%		Q(X) &= W_{0} + (W_{1} - I_{m})X + W_{2}X^{2} = 0, \label{eq:ex_8deg1}\\
%		R(X) &= W_{3} + (W_{4} - I_{m})X + W_{5}X^{2} = 0 \label{eq:ex_8deg2}
%	\end{align}
%	be given, where $W_{k}$ is a random nonnegative matrix which has null diagonal entries 
%	and positive off-diagonal entries such that $W_{k} \ones{m} = \frac13\ones{m}$ for $k = 0,1,2$,
%	$W_{3} \ones{m} = \frac{1}{2}\ones{m}$, and $W_{k} \ones{m} = \frac{1}{4}\ones{m}$ for $k = 4,5$.
%	Since $\dtyl I - \sum_{k=0}^{2}W_{k}$ and $\dtyl I - \sum_{k=3}^{5}W_{k}$ are singular irreducible $M$-matrices
%	and the coefficients satisfy Assumption \ref{ass:MPE},
%	\eqref{eq:ex_8deg1} and \eqref{eq:ex_8deg2} have the minimal positive solutions.
%	
%	We run the pure Newton algorithm(PNA) and the modified Newton algorithm(MNA) 300 times with $X_{0} = 0$, for \eqref{eq:ex_8deg1} and \eqref{eq:ex_8deg2} respectively.
%	All of the 300 experiments are run with different coefficients.
%	Averages of the iteration numbers are given in Table \ref{table:nocommonW}.
%	\begin{table}[h!]
%		\begin{tabular}{|c|c|c|}
%			\hline
%			&PNA&MNA\\ \hline
%			$Q(X) = 0$&26&10.99\\ \hline
%			$R(X) = 0$&7&7\\ \hline
%		\end{tabular}
%		\caption{\label{table:nocommonW}Averages of the iteration numbers}
%	\end{table}
%	
%	For \eqref{eq:ex_8deg1}, MNA finds the solution faster than PNA.
%	But, for \eqref{eq:ex_8deg2}, the iteration numbers of MNA and PNA are same.
%	Figure \ref{figure:ex2} shows that the solution $S_{Q}$ of \eqref{eq:ex_8deg1} is numerically non-simple and the solution $S_{R}$ of \eqref{eq:ex_8deg2} is simple.
%%	MNA is not efficient for simple solutions, i.e., MNA and PNA are not different for simple solutions.
%%	But, it is not a problem since iterations from MNA converge quadratically.
%%	In the experiment, for non-simple solutions, MNA lowers the iterations of PNA nearly in half.
%%	Therefore, we see that MNA is more efficient than PNA.
%	We can see from Table \ref{table:nocommonW} that MNA is significantly more efficient than PNA for finding non-simple solutions.
%%	\begin{figure}[t!]
%%		\begin{center}
%%			\resizebox{10cm}{8cm}{\includegraphics{ex2.pdf}}
%%			\caption{\label{figure:ex2}Relative errors of PNA iterations with the numerical solution $\hat{S}_{Q}$ and $\hat{S}_{R}$ from \eqref{eq:ex_8deg1} and \eqref{eq:ex_8deg2}, respectively}
%%		\end{center}
%%	\end{figure}
%\end{exam}

\section{Analysis for Nearly-Singular Points}
\begin{assumption}
	We assume that
	\begin{enumerate}
		\item $F(s) = 0$;
		\item $N_{1}$ is eigenspace for the smallest eigenvalue;
		\item $X = N_{1} \oplus X_{1}$ with a closed $X_{1}$;
		\item $F'[s]X_{1} = X_{1}$, $F'[s]N_{1} = N_{1}$;
		\item $P_{X_{1}}$ and $P_{N_{1}}$;
		\item $W_{\rho, \theta}(s) = \left\{ x \in X : \|x - s\| \leq \rho,
		\|P_{X_{1}}(x - s) \leq \theta\|P_{N_{1}}(x - s)\| \right\}$.
	\end{enumerate}
\end{assumption}

\begin{definition}
	\begin{enumerate}
		\item $\gamma(M, N) = \inf\{{\rm dist}(u, N) : u \in M, \|u\| = 1\}$.
	\end{enumerate}
\end{definition}

\begin{thm}\label{thm:main}
	Let
	\begin{enumerate}
		\item $N_{1}$ : 1-dim.;
		\item $F''[s]N_{1}N_{1} \cap X_{1} = \{0\}$;
		\item $\|F''[s]nx\| \geq c_{1}\|n\|\|x\|$ for all $n\in N_{1}, x \in X$ with $c_{1} > 0$.
	\end{enumerate}
	Then,
	\begin{enumerate}
		\item $\exists \rho > 0,~\theta > 0$ : $\exists F'[x]^{-1}$ for all $x(\neq s) 
		\in W_{\rho,\theta}(s)$;
		\item $\|F'[x]^{-1}\| \leq c_{2}\|x-s\|^{-1}$;
		\item $GW_{\rho,\theta}(s) \subset W_{\rho,\theta}(s)$ where $Gx \equiv x - F'[x]^{-1}F(x)$;
		\item $F^{-1}(0) \cap B_{\rho}(s) = s$;
		\item $x_{i} = Gx_{i-1}$ $\Rightarrow$ $x_{i} \rightarrow s$ 
		and $\|P_{X_{1}}(x_{i}-s)\|\leq c\|x_{i-1}-s\|^{2}$.
	\end{enumerate}
\end{thm}

\begin{proof}
	Let $\theta > 0$ and $\rho > 0$ be chosen.
	Since $F(x) = F(s) + F'[s](x-s) + \beta_{2}(x)$, 
	then $F'[x]^{-1}F(x) = F'[x]^{-1}F'[s]{\color{red}P_{X_{1}}(x-s)} + \beta_{1}(x) 
	= P_{X_{1}}(x-s)+\beta_{1}(x)$ for $x \neq s$ in $W$.
	{\color{red} Thus $Gx \rightarrow s$ as $x \rightarrow s$ in $W$.}
	Hence we define $Gs = s$.
	
	\begin{align*}
		F'[s]y &= F'[x_{0}]y + F''[x_{0}](s - x_{0})y + \frac12 F'''[x_{0}](s - x_{0})^{2}y + \cdots
	\end{align*}
	
	\begin{align*}
	F''[x_{0}](x_{0} - s)y &= F'[x_{0}]y - F'[s]y + \frac12 F'''[x_{0}](s - x_{0})^{2}y + \cdots
	\end{align*}
	
	\begin{align*}
	F''[x_{0}](x_{0} - s)P_{N_{1}}(x_{0} - s) &= (F'[x_{0}] - F'[s])P_{N_{1}}(x_{0} - s) + \beta_{3}
	\end{align*}
	
	Choose $x_{0} \in W$.
	Then $x_{1} = Gx_{0}$, the first Newton iterate, is defined and
	\begin{equation*}
		F'[x_{0}](x_{1} - s) = F'[x_{0}](x_{0} - s) - F(x_{0}).
	\end{equation*}
	Now
	\begin{equation*}
		F'[x_{0}]^{-1}F(x_{0}) = -(s-x_{0}) - F'[x_{0}]^{-1}\dfrac{F''[x_{0}]}{2}(s-x_{0})^{2} + \beta_{2}(x_{0}),
	\end{equation*}
	and so
	\begin{align*}
		x_{1} - s &= x_{0} - s - F'[x_{0}]^{-1}F(x_{0}) \\
		&= \frac12 F'[x_{0}]^{-1}F''[x_{0}](x_{0} - s)^{2} + \beta_{2}(x_{0}) \\
		&= \frac12 F'[x_{0}]^{-1}F''[x_{0}](x_{0} - s)P_{X_{1}}(x_{0} - s) \\
		&\qquad + \frac12 F'[x_{0}]^{-1}F''[x_{0}](x_{0} - s)P_{N_{1}}(x_{0} - s) + \beta_{2}(x_{0}) \\
		&= \frac12 F'[x_{0}]^{-1}F''[x_{0}](x_{0} - s)P_{X_{1}}(x_{0} - s) \\
		&\qquad + \frac12 F'[x_{0}]^{-1}(F'[x_{0}] - F'[s])P_{N_{1}}(x_{0} - s) + \beta_{2}(x_{0}) \\
		&= \frac12 F'[x_{0}]^{-1}F''[x_{0}](x_{0} - s)P_{X_{1}}(x_{0} - s) \\
		&\qquad + \frac12 P_{N_{1}}(x_{0} - s) - \frac12 F'[x_{0}]^{-1}F'[s]P_{N_{1}}(x_{0} - s) + \beta_{2}(x_{0}) \\
		&= \frac12 F'[x_{0}]^{-1}F''[x_{0}](x_{0} - s)P_{X_{1}}(x_{0} - s) \\
		&\qquad + \frac12 P_{N_{1}}(x_{0} - s) - \frac12 \l F'[x_{0}]^{-1}P_{N_{1}}(x_{0} - s) + \beta_{2}(x_{0}) 
	\end{align*}
	
	\begin{align*}
		P_{X_{1}}(x_{1} - s) &= \frac12 (P_{X_{1}}F'[x_{0}]^{-1})F''[x_{0}](x_{0} - s)P_{X_{1}}(x_{0} - s) \\
		&\qquad - \frac12 \l P_{X_{1}}F'[x_{0}]^{-1}P_{N_{1}}(x_{0} - s) + \beta_{2}(x_{0}) 
	\end{align*}
	
	\begin{align*}
	\|P_{X_{1}}(x_{1} - s)\| &\leq c^*\|x_{0} - s\|^{2} + |\l|c^{**}\|x_{0} - s\| \\
	\dfrac{\|P_{X_{1}}(x_{1} - s)\|}{\|x_{0} - s\|} & \leq c^{*}\|x_{0} - s\| + |\l|c^{**}
	\end{align*}
\end{proof}

\begin{lem}
	\begin{enumerate}
		\item There exists a constant $c_{3} > 0$ such that $\|F'[x]y\| \geq c_{3}\|y\|$ holds 
		for all $y \in X$ and $x \in B_{\rho}(s)$.
		{\color{red}\item There exists a constant $c_{4} > 0$ so that 
			$\|(F'[x] - \l I)m\| \geq c_{4}\|x-s\|\|m\|$
		holds for all $m \in N_{1}$ and all $x \in B_{\rho}(s)$.\label{lem:enu:2}}
	\end{enumerate}
\end{lem}

\begin{proof}
	\eqref{lem:enu:2}
	\begin{align*}
		F'[x]m &= F'[s]m + F''[s]m(x - s) + \frac12 F'''[s]m(x - s)(x - s) + O(\|m\|\|x - s\|^3)\\
		&= \l m + F''[s]m(x - s) + O(\|m\|\|x - s\|^2)\\
		&= \l m + O(\|m\|\|x - s\|)
	\end{align*}
	\begin{align*}
		F'[x]m - \l m = O(\|m\|\|x - s\|)\\
		(F'[x] - \l I) m = O(\|m\|\|x - s\|)
	\end{align*}
\end{proof}

\begin{lemma}
	There exist positive constants $c_{5}$, $\rho$, and $\theta$ such that $\gamma(F'[x]N_{1}, F'[x]X_{1}) \geq c_{5}$ for $x \neq s$ in $W_{\rho, \theta}(s)$.
\end{lemma}

{\color{gray}
\begin{proof}
	Fix $\rho > 0$. Suppose the conclusion is false.
	Then we have sequences $\{x_{i}\} \in B_{\rho}(s)$, $\{m_{i}\} \in N_{1}$, and $\{y_{i}\} \in X_{1}$
	such that $\|F'[x_{i}]m_{i}\| = 1$, $\|F'[x_{i}]m_{i} - F'[x_{i}]y_{i}\| \equiv \e_{i} \rightarrow 0$, 
	and $\|P_{X_{i}}(x_{i} - s)\| \leq \theta_{i}\|P_{N_{1}}(x_{i} - s)\|$ with $\theta_{i} \rightarrow 0$.
	{\color{red}Note for $\rho$ small $F'[x]N_{1} \neq \{0\}$, $x \neq s$ in $B_{\rho}(s)$.}
	Anyway, $\gamma(F'[x]N_{1}, F'[x]X_{1})$ is defined.
	Note also that there exist $a_{1}, a_{2} > 0$ such that $a_{1} \leq \|y_{i}\| \leq a_{2}$.
	{\color{red}Since $F''[s]N_{1}N_{1}$ is 1-dim. of $X$ such that $F''[s]N_{1}N_{1} \cap X_{1} = \{0\}$,
	$\|w - P_{X_{1}}w\| \geq \alpha \|w\|$ for some $\alpha > 0$ and all $w$ in $F''[s]N_{1}N_{1}$.}
	We finally note that 
	{\color{red}since $F'[x_{i}]m_{i} = F'[s]m_{i} + F''[s](x_{i} - s)m_{i} + \beta_{2}(x_{i})m_{i}$,} then
	\begin{align*}
		F'[x_{i}]m_{i} &= F'[s]m_{i} + F''[s](x_{i} - s)m_{i} + \beta_{2}(x_{i})m_{i}\\
		F'[x_{i}]m_{i} - \beta_{2}(x_{i})m_{i} - \l m_{i} &= F''[s](x_{i} - s)m_{i} \\
		\|F'[x_{i}]m_{i} - \beta_{2}(x_{i})m_{i} - \l m_{i}\| &= \|F''[s](x_{i} - s)m_{i}\| \\
		\|F'[x_{i}]m_{i}\| + c_{6}\|x_{i}-s\|^{2}\|m_{i}\| + |\l|\|m_{i}\| 
		& \geq \|F'[x_{i}]m_{i}\| + \|\beta_{2}(x_{i})m_{i}\| + |\l|\|m_{i}\| \\
		&\geq \|F''[s](x_{i} - s)m_{i}\| \\
		&\geq c_{1}\|x_{i} - s\|\|m_{i}\|
	\end{align*}
	
	\begin{equation}
		1 \geq c_{1}\|x_{i} - s\|\|m_{i}\| - c_{6}\|x_{i} - s\|^{2}\|m_{i}\| - |\l|\|m_{i}\|
	\end{equation}
	and
	\begin{equation}
		1 \leq |\l|\|m_{i}\| + \|F''[s]\|\|x_{i} - s\|\|m_{i}\| + c_{6}\|x_{i} - s\|^{2}\|m_{i}\|,
	\end{equation}
	{\color{red}if $\|F''[s]\|$ is bounded}.
	Thus for some constants $a_{3}$ and $a_{4}$,
	\begin{equation}
		0 < a_{3} \leq \|x_{i} - s\|\cdot\|m_{i}\| \leq a_{4}
	\end{equation}
	holds for $\rho$ sufficiently small.
	
	Define $w_{i} = F''[s]P_{N_{1}}(x_{i} - s)m_{i}$. Then,
	\begin{align*}
		\alpha c_{1} \|P_{N_{1}}(x_{i} - s)\|\cdot\|m_{i}\| 
		& \leq \|w_{i} - P_{X_{1}}w_{i}\| \leq c\|w_{i} - F'[s]y_{i}\|\\
		& \leq c \|F''[s](x_{i}-s)m_{i} - F'[s]y_{i}\| + c\|F''[s]P_{X_{1}}(x_{i} - s)m_{i}\|\\
		& \leq c\|F''[s](x_{i}-s)m_{i} - F'[x_{i}]y_{i}\| + c\|F'[x_{i}]y_{i}-F'[s]y_{i}\|\\
		& \qquad\qquad +c \|F''[s]P_{X_{1}}(x_{i} - s)m_{i}\| \numberthis \label{eq:longeq}\\
		& \leq c\e_{i} + c c_{6}\|x_{i} - s\|^{2}\|m_{i}\| + \|\beta_{1}(x_{i})\|a_{2}c\\
		& \qquad\qquad + c\|F''[s]\|\theta_{i}\|P_{N_{1}}(x_{i} - s)\|\cdot\|m_{i}\|,
	\end{align*}
	where $\|P_{N_{1}}z\| \leq c\|z\|$. Note that $\|P_{X_{1}}(x_{i} - s)\| \leq c\theta_{i}\|x_{i}-s\|$
	and $\|P_{N_{1}}(x_{i} - s)\| \geq (1 - c\theta_{i})\|x_{i} - s\|$.
	Substituting into \eqref{eq:longeq} and dividing by $\|x_{i}-s\|\cdot\|n_{i}\|$,
	we have
	\begin{equation}
		\alpha c_{1}(1-c\theta_{i}) \leq c\e_{i}a_{3}^{-1}
		+ cc_{6}\|x_{i} - s\| + c_{7}a_{3}^{-1}\|x_{i} - s\| + c_{5}\theta_{i}.
	\end{equation}
	Thus for $i$ sufficiently large, we have $\alpha c_{1} \leq 2(cc_{6}+c_{7}a_{3}^{-1})\|x-s\|$.
	Thus for $\rho$ sufficiently small, we have a contradiction, completing the proof.
\end{proof}}

\section{Note}

\begin{align*}
Q(X) &= A_{2}X^{2} + A_{1}X + A_{0}
\end{align*}

\bibliographystyle{plain}
\bibliography{SHSeo}

\end{document}